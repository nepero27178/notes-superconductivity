\chapter{The BCS theory}\chaptertoc{}\label{chap:the bcs theory}

It is time to develop formally and completely the theory by Bardeen, Cooper and Schrieffer. We saw in the last chapter how electrons form Cooper pairs, because apparently in superconductors charges flow coupled. The question now is: why is the coupling of electrons (a rather weak one, also) necessary for the exotic phenomena of superconductivity, like resistanceless flow of charge and Meissner effect?

\section{BCS theory}

This section is devoted to the formal, quantum-mechanical treatment of BCS theory. The analysis is brought with two complementary methods.

\subsection{The BCS hamiltonian and ground state}

From last chapter we know that the phonon-mediated effective hamiltonian is given by
\[
	\hat{\bm H} = \sum_{\mathbf{k}} \epsilon_{\mathbf{k}} \lrS{\hat{c}_{\mathbf{k}\uparrow}^\dagger \hat{c}_{\mathbf{k}\uparrow} + \hat{c}_{\mathbf{k}\downarrow}^\dagger \hat{c}_{\mathbf{k}\downarrow}} + \sum_{\mathbf{k}\mathbf{k}'} V_{\mathbf{k}-\mathbf{k}'} \lrS{\hat{c}_{\mathbf{k}\uparrow}^\dagger \hat{c}_{\mathbf{k}'\uparrow}} \lrS{\hat{c}_{-\mathbf{k}\downarrow}^\dagger \hat{c}_{-\mathbf{k}'\downarrow}}
\]
We also know that the Fermi Sea,
\[
	\ket{\mathrm{F}} \equiv \bigotimes_{\abs{\mathbf{k}} < k_F} \hat{c}_{\mathbf{k}\uparrow}^\dagger \hat{c}_{\mathbf{k}\downarrow}^\dagger \ket{0} = \bigotimes_{\abs{\mathbf{k}} < k_F} \hat{c}_{\mathbf{k}\uparrow}^\dagger \hat{c}_{-\mathbf{k}\downarrow}^\dagger \ket{\Omega}
\]
with $\ket{\Omega}$ the state with $0$ electrons, is an unstable state. This means, as it is obvious now, that the Fermi Sea configuration is not the ground state of the system. We know that the ground state will be in some measure populated by a mixture of Cooper pairs; note that, even if we have a general idea of how a \textit{single} Cooper pair looks like, we have no way of predicting the real correlated state with multiple pairs. 

The key idea is: the interaction involves a small portion of the Fermi sphere, mainly a thin shell around the surface. The number of particles is a physically conserved quantity; however by defining the BCS ground state $\ket{\Psi}$ as
\begin{equation}\label{eq:bcs ground state general form}
	\ket{\Psi} \equiv \bigotimes_\mathbf{k} \lrS{u_\mathbf{k} + v_\mathbf{k} \hat{c}_{\mathbf{k}\uparrow}^\dagger \hat{c}_{-\mathbf{k}\downarrow}^\dagger} \ket{\Omega}
	\qq{with}
	u_\mathbf{k}, v_\mathbf{k} \in \C
\end{equation}
we shall not make a big mistake if $u_\mathbf{k}$ vanishes rapidly enough inside the Fermi sphere and $v_\mathbf{k}$ outside. Such condition is similar to the Fermi sphere configuration,
\[
	u_\mathbf{k} = \theta\lrR{k_F - \abs{\mathbf{k}}}
	\quad\qq{and}\quad
	v_\mathbf{k} = \theta\lrR{\abs{\mathbf{k}} - k_F}
\]
that can be taken as its limiting case.
Defined as it is, $\ket{\Psi}$ is a superposition of many states with different number of particles,
\[
\begin{aligned}
	\ket{\Psi} &= \prod_\mathbf{k} u_\mathbf{k} \ket{\Omega} \\
	&+ \sum_{\mathbf{k}_1} \prod_{\mathbf{k} \neq \mathbf{k}_1} u_\mathbf{k} v_{\mathbf{k}_1} \lrS{\hat{c}_{\mathbf{k}_1\uparrow}^\dagger \hat{c}_{-\mathbf{k}_1\downarrow}^\dagger} \ket{\Omega} \\
	&+ \sum_{\mathbf{k}_1 \mathbf{k}_2} \prod_{\mathbf{k} \neq \mathbf{k}_1, \mathbf{k}_2} u_\mathbf{k} v_{\mathbf{k}_1} \lrS{\hat{c}_{\mathbf{k}_1\uparrow}^\dagger \hat{c}_{-\mathbf{k}_1\downarrow}^\dagger} v_{\mathbf{k}_2} \lrS{\hat{c}_{\mathbf{k}_2\uparrow}^\dagger \hat{c}_{-\mathbf{k}_2\downarrow}^\dagger} \ket{\Omega}
\end{aligned}
\]
The first term has zero particles; the second has one pair; the third has two pairs, and so on. For our description to be coherent, we expect in thermodynamic limit the number of particles to localize, which is, to negligibly fluctuate around its mean value. Next section deals with this problem.

\subsection{How many particles?}

As said, we want the mean number of particles $\ev{\hat{N}}$ in the ground state to be a well-defined quantity, at least in the thermodynamic limit. So, consider the number operators,
\[
	\hat{N}_\uparrow \equiv \sum_{\mathbf{k}}
		\hat{c}_{\mathbf{k}\uparrow}^\dagger \hat{c}_{\mathbf{k}\uparrow}
	\qquad
	\hat{N}_\downarrow \equiv \sum_{\mathbf{k}}
		\hat{c}_{\mathbf{k}\downarrow}^\dagger \hat{c}_{\mathbf{k}\downarrow}
	\qquad
	\hat{N} \equiv \hat{N}_\uparrow + \hat{N}_\downarrow
\]
which count the number of particles for a given state. Simple calculations lead us to
\[
	\mel{\Psi}{\hat{N}_\uparrow}{\Psi} = \sum_{\mathbf{k}} \abs{v_\mathbf{k}}^2
	\qquad
	\mel{\Psi}{\hat{N}_\downarrow}{\Psi} = \sum_{\mathbf{k}} \abs{v_{-\mathbf{k}}}^2
\]
This result is rather obvious, once seen the form of Eq.~\eqref{eq:bcs ground state general form}. The parameter $v_\mathbf{k}$ is the probability amplitude for the pair occupation of the states $\ket{\mathbf{k}\uparrow}$ and $\ket{-\mathbf{k}\downarrow}$.
We impose spin balance, so that $\abs{v_{-\mathbf{k}}}^2 = \abs{v_\mathbf{k}}^2$. Notice that requiring $v_{-\mathbf{k}} = v_\mathbf{k}$ implies the parameter to be real and the above condition to be satisfied. We will make that assumption. Thus, we have
\[
	\ev{\hat{N}} = 2\sum_{\mathbf{k}} \abs{v_\mathbf{k}}^2
\]

To get the number fluctuations, we need to extract
\[
	\ev{\hat{N}^2} = \big\langle \hat{N}_\uparrow^2 \big\rangle + \ev{\hat{N}_\uparrow \hat{N}_\downarrow} + \ev{\hat{N}_\downarrow \hat{N}_\uparrow} + \big\langle \hat{N}_\downarrow^2 \big\rangle
\]
that turns out to be
\[
	\ev{\hat{N}^2} = 4 \sum_{\mathbf{k} \neq \mathbf{k}'} \abs{v_\mathbf{k}}^2 \abs{v_{\mathbf{k}'}}^2 + 2 \ev{\hat{N}}
\]
and this implies
\[
\begin{aligned}
	\frac{\sqrt{\ev{\hat{N}^2} - \ev{\hat{N}} \vphantom{\hat{N}}^2}}{\ev{\hat{N}}} &= \frac{\displaystyle
		\sqrt{2\ev{\hat{N}} + 4\sum_{\mathbf{k}\neq\mathbf{k}'} \abs{v_\mathbf{k}}^2 \abs{v_{\mathbf{k}'}}^2 - 4\sum_{\mathbf{k}} \abs{v_\mathbf{k}}^2 \sum_{\mathbf{k}'} \abs{v_{\mathbf{k}'}}^2}
	}{ \ev{\hat{N}}	} \\
	&= \frac{\displaystyle
		\sqrt{2\ev{\hat{N}} - 4\sum_\mathbf{k} \abs{v_\mathbf{k}}^4}
	}{ \ev{\hat{N}}	} < \sqrt{\frac{2}{\ev{\hat{N}}}}
\end{aligned}
\]
thus in thermodynamic limit the number of particles is a well-defined quantity.


Now: to extract the BCS value of $u_\mathbf{k}$, $v_\mathbf{k}$ two ways are the most commonly used. Both are interesting, so we shall explore them separately.

\subsection{The variational method}\label{subsec:the variational method}

The key idea is: the energy is a functional of $u_\mathbf{k}$, $v_\mathbf{k}$. To extract the energy we need to minimize the functional with respect to parameters variations. To simplify, we expect one parameter to be ``fictitious'' -- in the sense that due to normalization of $\ket{\Psi}$ some relation between $u_\mathbf{k}$ and $v_\mathbf{k}$ must exist, limiting our variational approach to a single parameter. So we compute $\braket{\Psi}{\Psi}$,
\[
\begin{aligned}
	\braket{\Psi}{\Psi} &= \bigotimes_\mathbf{k} \bra{\Omega} \lrS{
		\vphantom{c^\dagger}
		u_\mathbf{k}^* + v_\mathbf{k}^* \hat{c}_{-\mathbf{k}\downarrow} \hat{c}_{\mathbf{k}\uparrow}
	} \bigotimes_{\mathbf{k}'} \lrS{
		u_{\mathbf{k}'} + v_{\mathbf{k}'} \hat{c}_{\mathbf{k}'\uparrow}^\dagger \hat{c}_{-\mathbf{k}'\downarrow}^\dagger
	} \ket{\Omega} \\
	&= \bigotimes_\mathbf{k} \bra{\Omega} \Big[
		\abs{u_\mathbf{k}}^2 + u_\mathbf{k}^* v_\mathbf{k} \hat{c}_{\mathbf{k}\uparrow}^\dagger \hat{c}_{-\mathbf{k}\downarrow}^\dagger + u_\mathbf{k} v_\mathbf{k}^* \hat{c}_{-\mathbf{k}\downarrow} \hat{c}_{\mathbf{k}\uparrow} \\
	&\hspace{12.12em} + \abs{v_\mathbf{k}}^2 \hat{c}_{-\mathbf{k}\downarrow} \hat{c}_{\mathbf{k}\uparrow} \hat{c}_{\mathbf{k}\uparrow}^\dagger \hat{c}_{-\mathbf{k}\downarrow}^\dagger
	\Big] \ket{\Omega} \\
	&= \bigotimes_\mathbf{k} \lrS{
		\vphantom{c^\dagger}
		\abs{u_\mathbf{k}}^2 + \abs{v_\mathbf{k}}^2
	} \imp 1
\end{aligned}
\]
since mixed terms vanish and having used fermionic commutation rules. The above condition is solved by
\[
	\forall\mathbf{k}
	\quad\colon\quad
	\abs{u_\mathbf{k}}^2 + \abs{v_\mathbf{k}}^2 = 1
\]
We use $u_\mathbf{k} , v_\mathbf{k} \in \R$ {\color{red}Why?}. This relations allows us to define a single parameter $\theta_\mathbf{k}$,
\[
	u_\mathbf{k} \equiv \cos\theta_\mathbf{k}
	\quad\qq{and}\quad
	v_\mathbf{k} \equiv \sin\theta_\mathbf{k}
\]
Now we want to find some expression for the functional $E\lrS{\theta_\mathbf{k}}$. We consider shifting the energies up to $\epsilon_F$, thus considering the operator
\[
	\hat{\bm H} - \epsilon_F \hat{N}
	\quad\qq{with}\quad
	\hat{N} \equiv \sum_{\mathbf{k}} \lrS{
		\hat{c}_{\mathbf{k}\uparrow}^\dagger \hat{c}_{\mathbf{k}\uparrow} + \hat{c}_{\mathbf{k}\downarrow}^\dagger \hat{c}_{\mathbf{k}\downarrow}
	}
\]
and define the functional as
\[
\begin{aligned}
	E\lrS{\theta_\mathbf{k}} &\equiv \mel{\Psi}{\lrS{\hat{\bm H} - \epsilon_F \hat{N}}}{\Psi} \\
	&= \sum_{\mathbf{k}} \xi_{\mathbf{k}} \mel{\Psi}{\lrS{\hat{c}_{\mathbf{k}\uparrow}^\dagger \hat{c}_{\mathbf{k}\uparrow} + \hat{c}_{\mathbf{k}\downarrow}^\dagger \hat{c}_{\mathbf{k}\downarrow}}}{\Psi} \\ 
	&\hspace{5em} + \sum_{\mathbf{k}\mathbf{k}'} V_{\mathbf{k}-\mathbf{k}'} \mel{\Psi}{\lrS{\hat{c}_{\mathbf{k}\uparrow}^\dagger \hat{c}_{-\mathbf{k}\downarrow}^\dagger} \lrS{\vphantom{c^\dagger} \hat{c}_{-\mathbf{k}'\downarrow} \hat{c}_{\mathbf{k}'\uparrow}}}{\Psi}
\end{aligned}
\]
with $\xi_{\mathbf{k}} \equiv \epsilon_{\mathbf{k}} - \epsilon_F$, and where some fermionic rules have been used. Some straightforward calculations leads us to
\[
\begin{aligned}
	E\lrS{\theta_\mathbf{k}} &= 2 \sum_{\mathbf{k}} \xi_{\mathbf{k}} \abs{v_\mathbf{k}}^2 + \sum_{\mathbf{k}\mathbf{k}'} V_{\mathbf{k}-\mathbf{k}'} \lrS{v_\mathbf{k}^* u_\mathbf{k}} \lrS{u_{\mathbf{k}'}^* v_{\mathbf{k}'}} \\
	&= 2 \sum_{\mathbf{k}} \xi_{\mathbf{k}} \sin^2\theta_\mathbf{k} + \sum_{\mathbf{k}\mathbf{k}'} V_{\mathbf{k}-\mathbf{k}'}
	\sin\theta_\mathbf{k} \cos\theta_\mathbf{k}
	\cos\theta_{\mathbf{k}'} \sin\theta_{\mathbf{k}'} \\
	&= 2 \sum_{\mathbf{k}} \xi_{\mathbf{k}} \sin^2\theta_\mathbf{k} + \frac{1}{4} \sum_{\mathbf{k}\mathbf{k}'} V_{\mathbf{k}-\mathbf{k}'}
	\sin2\theta_\mathbf{k} \sin2\theta_{\mathbf{k}'}
\end{aligned}
\]
where we used $2 \sin\theta\cos\theta = \sin2\theta$.
Then:
\[
\begin{aligned}
	0 &\imp \pdv{}{\theta_\mathbf{k}} E\lrS{\theta_\mathbf{k}} \\
	&= 2 \xi_\mathbf{k} \sin2\theta_\mathbf{k} + \frac{1}{2} 
	\cos2\theta_\mathbf{k} \sum_{\mathbf{k}'} V_{\mathbf{k}-\mathbf{k}'} \sin2\theta_{\mathbf{k}'} 
\end{aligned}
\]
since $\partial_\theta \sin^2 \theta = \sin2\theta$. We define
\begin{equation}\label{eq:delta variational definition}
	\Delta_\mathbf{k} \equiv - \frac{1}{4} 
	\sum_{\mathbf{k}'} V_{\mathbf{k}-\mathbf{k}'} \sin2\theta_{\mathbf{k}'}
	\quad\implies\quad
	\xi_\mathbf{k} \sin2\theta_\mathbf{k} = \Delta_\mathbf{k} \cos2\theta_\mathbf{k}
\end{equation}
Expanding the above result and taking its square it is easy to get to the equation
\[
	4 \lrR{\xi_\mathbf{k} + \Delta_\mathbf{k}}^2 \cos^4\theta_\mathbf{k} - 4 \lrR{\xi_\mathbf{k} + \Delta_\mathbf{k}}^2 \cos^2\theta_\mathbf{k} + \Delta_\mathbf{k}^2 = 0
\]
or, substituting
\[
	\abs{v_\mathbf{k}}^4 - \abs{v_\mathbf{k}}^2 + \frac{\Delta_\mathbf{k}^2}{4 \lrR{\xi_\mathbf{k} + \Delta_\mathbf{k}}^2} = 0
\]
Implementing the condition
\[
	\lim_{\xi_\mathbf{k} \gg \Delta_\mathbf{k}} v_\mathbf{k} \imp 0
\]
which guarantees that the ground state has no population for states very distant from the Fermi surface, we get the solution
\begin{equation}\label{eq:v variational solution}
	\abs{v_\mathbf{k}}^2 = \frac{1}{2} \lrR{1 - \frac{\xi_\mathbf{k}}{\sqrt{\xi_\mathbf{k}^2 + \Delta_\mathbf{k}^2}}} 
\end{equation}
and due to normalization
\begin{equation}\label{eq:u variational solution}
	\abs{u_\mathbf{k}}^2 = \frac{1}{2} \lrR{1 + \frac{\xi_\mathbf{k}}{\sqrt{\xi_\mathbf{k}^2 + \Delta_\mathbf{k}^2}}} 
\end{equation}

\begin{figure}
	\centering
	\def\ShellWidth{0.25}
\begin{tikzpicture}
	\begin{axis}[
		axis x line=center,
		axis y line=center,
		xlabel={$\xi_\mathbf{k}$},
		ylabel=\empty,
		xlabel style=right,
		ylabel style=\empty,
		xtick={-\ShellWidth,\ShellWidth},
		ytick={1},
		xticklabels={$-\Delta$,$\Delta$},
		yticklabel={$1$},
		xticklabel style=\empty,
		yticklabel style=\empty,
		xmin=-1.1, xmax=1.1,
		ymin=-0.1, ymax=1.1
		]
			
		\addplot[color=lev,domain=-1:1,smooth]
			{1/2*(1+(x/sqrt(x^2+\ShellWidth^2)))};
		\node[color=lev,anchor=center] at (0.8,0.85) 
			{$\abs{u_\mathbf{k}}^2$};
		\addplot[color=red,domain=-1:1,smooth]
			{1/2*(1-(x/sqrt(x^2+\ShellWidth^2)))};
		\node[color=red,anchor=center] at (-0.8,0.85) 
			{$\abs{v_\mathbf{k}}^2$};
	\end{axis}
\end{tikzpicture}
	\caption{Plot of the variational solutions for the problem parameters $u_\mathbf{k}$, in Eq.~\eqref{eq:u variational solution}, and $v_\mathbf{k}$, in Eq.~\eqref{eq:v variational solution}. This plot was realized using a constant Ansatz, $\Delta_\mathbf{k} = \Delta$. This plot is in general reasonable but not completely coherent with the theory; instead, the plot in Fig.~\ref{fig:actual variational solutions} is coherent with.}
	\label{fig:variational solutions}
\end{figure}

In Fig.~\ref{fig:variational solutions} the two solutions are plotted as functions of the energy $\xi_\mathbf{k}$ in the special case $\Delta_\mathbf{k} = \Delta$. We do not expect this to be the actual solution, but it is a good starting point to understand the general behavior of these functions. As evident, deep inside the sphere (for $\xi_\mathbf{k}\le0$) we have $v_\mathbf{k} \simeq 1$, and $u_\mathbf{k} \simeq 0$, leading to a state similar to the non-interacting perfect Fermi sphere. Far outside the situation is the opposite, $v_\mathbf{k} \simeq 0$, and $u_\mathbf{k} \simeq 1$, which correctly means that far states are not populated. Overall the state is pretty similar to the Fermi sphere, with significant variations only in the energy range
\[
	- \Delta < \xi_\mathbf{k} < \Delta
\]

\subsection{Self-consistency equation}

We aim to find an expression for $\Delta_\mathbf{k}$, using the found solutions. Substituting the obtained solutions in the definition of $\Delta_\mathbf{k}$, we get
\[
\begin{aligned}
	\Delta_\mathbf{k} &\equiv - \frac{1}{4} 
	\sum_{\mathbf{k}'} V_{\mathbf{k}-\mathbf{k}'} \sin2\theta_{\mathbf{k}'} \\
	&= - \frac{1}{2} 
	\sum_{\mathbf{k}'} V_{\mathbf{k}-\mathbf{k}'} v_{\mathbf{k}'} u_{\mathbf{k}'} \\
	&= - \frac{1}{2} 
	\sum_{\mathbf{k}'} V_{\mathbf{k}-\mathbf{k}'} \lrS{\frac{1}{4} \lrR{1 - \frac{\xi_{\mathbf{k}'}}{\sqrt{\xi_{\mathbf{k}'}^2 + \Delta_{\mathbf{k}'}^2}}} \lrR{1 + \frac{\xi_{\mathbf{k}'}}{\sqrt{\xi_{\mathbf{k}'}^2 + \Delta_{\mathbf{k}'}^2}}} }^{1/2} \\
	&= - \frac{1}{4} \sum_{\mathbf{k}'} V_{\mathbf{k}-\mathbf{k}'} \frac{\Delta_{\mathbf{k}'}}{\sqrt{\xi_{\mathbf{k}'}^2 + \Delta_{\mathbf{k}'}^2}}
\end{aligned}
\]
This is the so-called \textbf{self-consistency equation}.
Consider now the potential analyzed in Sec.~\ref{subsec:bound states, considering statistics}. The phonon effective potential of Sec.~\ref{subsec:a few assumptions, and cooper pairs} is well approximated by
\[
	V_{\mathbf{k}-\mathbf{k}'} = - V_0 A\lrR{\mathbf{k}} A\lrR{\mathbf{k}'}
\]
with $A$, as always, the characteristic function of the shell. Then
\[
	\Delta_\mathbf{k} = \frac{V_0 A\lrR{\mathbf{k}}}{4} \sum_{\mathbf{k}'} A\lrR{\mathbf{k}'} \frac{\Delta_{\mathbf{k}'}}{\sqrt{\xi_{\mathbf{k}'}^2 + \Delta_{\mathbf{k}'}^2}}
\]
The right part of the equation is zero for $\mathbf{k}$ outside the interaction shell, and constant inside. Then
\[
	\Delta_\mathbf{k} = \Delta A\lrR{\mathbf{k}}
\]
with $\Delta$ a constant term. It follows, inside the shell,
\[
	\Delta = \frac{V_0}{4} \sum_{\abs{\xi_\mathbf{k}}<\delta\epsilon^\star} \frac{\Delta}{\sqrt{\xi_\mathbf{k}^2 + \Delta^2}}
\]
with $\delta\epsilon^\star = \hbar\omega_D$, as explained in Sec.~\ref{subsec:a few assumptions, and cooper pairs}. Then, converting the sum into an energy integral and approximating the density of states constant as in Sec.~\ref{subsec:bound states, considering statistics}, we get
\[
	\Delta \simeq \frac{\rho_0 V_0}{4} \int_{-\hbar\omega_D}^{\hbar\omega_D} d\xi \frac{\Delta}{\sqrt{\xi^2 + \Delta^2}} = \frac{\rho_0 V_0}{2} \int_0^{\hbar\omega_D} d\xi \frac{\Delta}{\sqrt{\xi^2 + \Delta^2}}
\]
Then, changing variable $s=\xi/\Delta$, we recognize the derivative of $\sinh^{-1} s$,
\[
	1 = \frac{\rho_0 V_0}{2} \int_0^{\hbar\omega_D/\Delta} ds \frac{\Delta}{\sqrt{1 + s^2}} = \frac{\rho_0 V_0}{2} \sinh^{-1} \lrR{\frac{\hbar\omega_D}{\Delta}}
\]
which implies:
\[
	\Delta = \frac{\hbar\omega_D}{\displaystyle
		\sinh\lrR{\frac{2}{\rho_0 V_0}}
	}
\]

Now: since
\[
	\lim_{x\to+\infty} \sinh x = \lim_{x\to+\infty} \frac{e^x - e^{-x}}{2} = \lim_{x\to+\infty} \frac{e^x}{2}
\]
and we have seen that the potential $V_0$ expressed by phonon mediation is weak, we may approximate
\[
	\sinh\lrR{\frac{2}{\rho_0 V_0}} \simeq \frac{e^{2/\rho_0 V_0}}{2}
\]
which finally gives
\[
	\Delta = 2 \hbar\omega_D e^{-2/\rho_0 V_0}
\]
Familiar? That is precisely the binding energy of the Cooper pair of Sec.~\ref{subsec:a few assumptions, and cooper pairs}. We omit for a second more comments about the meaning of $\Delta$, which becomes clear enough in the following section.

Now: since $\Delta_\mathbf{k} = \Delta A\lrR{\mathbf{k}}$, we have
\begin{equation}\label{eq:u actual variational solution}
	\abs{u_\mathbf{k}}^2 = \begin{cases}
		0 &\hphantom{-\hbar\omega <{}}\xi_\mathbf{k} < -\hbar\omega_D\\
		\displaystyle\frac{1}{2} \lrR{1 + \frac{\xi_\mathbf{k}}{\sqrt{\xi_\mathbf{k}^2 + \Delta^2}}} &-\hbar\omega <\xi_\mathbf{k} < \hbar\omega_D\\
		1 &\hphantom{-\hbar\omega <{}}\xi_\mathbf{k} > \hbar\omega_D
	\end{cases}
\end{equation}
and
\begin{equation}\label{eq:v actual variational solution}
	\abs{v_\mathbf{k}}^2 = \begin{cases}
		1 &\hphantom{-\hbar\omega <{}}\xi_\mathbf{k} < -\hbar\omega_D\\
		\displaystyle\frac{1}{2} \lrR{1 - \frac{\xi_\mathbf{k}}{\sqrt{\xi_\mathbf{k}^2 + \Delta^2}}} &-\hbar\omega <\xi_\mathbf{k} < \hbar\omega_D\\
		0 &\hphantom{-\hbar\omega <{}}\xi_\mathbf{k} > \hbar\omega_D
	\end{cases}
\end{equation}
Notice that $\Delta < 2\hbar\omega_D$, but in principle it is not guaranteed $\Delta < \hbar\omega_D$. Check Fig.~\ref{fig:actual variational solutions}: the above functions are there plotted, with the arbitrary choice $\Delta = \hbar\omega_D/2$. Such choice was made for reasons of graphic clarity, but it is to be expected as more physical $\Delta \ll \hbar\omega_D$: for such regime the red and blue plots become more continuous and reasonable. Don't take too seriously the discontinuities: those are generated by the approximation $\Delta_\mathbf{k} = \Delta A(\mathbf{k})$.

\begin{figure}
	\centering
	\def\ShellWidth{1.5}
\def\BindingEnergy{\ShellWidth/6}
\begin{tikzpicture}
	\begin{axis}[
		axis x line=center,
		axis y line=center,
		axis equal image,
		xlabel={$\xi_\mathbf{k}$},
		ylabel=\empty,
		xlabel style=right,
		ylabel style=\empty,
		xtick={-\ShellWidth,-\BindingEnergy,\BindingEnergy,\ShellWidth},
		ytick={1},
		xticklabels={$-\hbar\omega_D\quad$,$-\Delta$,$\Delta$,$\quad\hbar\omega_D$},
		yticklabel={$1$},
		xticklabel style=\empty,
		yticklabel style={anchor=south east},
		xmin=-\ShellWidth*1.2, xmax=\ShellWidth*1.2,
		ymin=-0.1, ymax=1.2,
		width=\textwidth
		]
		
		% Interaction region
		\fill[color=lev!30,opacity=0.5] 
			(axis cs:-\ShellWidth,0) rectangle (axis cs:\ShellWidth,1);
		
		% Blue plot
		\addplot[color=lev,domain=-\ShellWidth*1.1:-\ShellWidth,smooth]
			{0};
		\addplot[color=lev,domain=-\ShellWidth:\ShellWidth,smooth]
			{1/2*(1+(x/sqrt(x^2+\BindingEnergy^2)))};
		\addplot[color=lev,domain=\ShellWidth:\ShellWidth*1.1,smooth]
			{1};
		\node[color=lev,anchor=center] at (\ShellWidth*0.8,0.85) 
			{$\abs{u_\mathbf{k}}^2$};
		
		% Red plot
		\addplot[color=red,domain=-\ShellWidth*1.1:-\ShellWidth,smooth]
			{1};
		\addplot[color=red,domain=-\ShellWidth:\ShellWidth,smooth]
			{1/2*(1-(x/sqrt(x^2+\BindingEnergy^2)))};
		\addplot[color=red,domain=\ShellWidth:\ShellWidth*1.1,smooth]
			{0};
		\node[color=red,anchor=center] at (-\ShellWidth*0.8,0.85) 
			{$\abs{v_\mathbf{k}}^2$};
	\end{axis}
\end{tikzpicture}
	\caption{Plot of the amplitudes $\abs{u_\mathbf{k}}^2$ and $\abs{v_\mathbf{k}}^2$ as reported in Eqns.~\eqref{eq:u actual variational solution} and \eqref{eq:v actual variational solution}. The shaded region $\abs{\xi_\mathbf{k}} < \hbar\omega_D$ is the interaction region, out of which the solution is correctly given by the Fermi state. In this plot we arbitrarily set $\Delta = \hbar\omega_D/2$, inspired by the idea of weakness of the attracting potential. In general, for $\Delta \ll \hbar\omega_D$, as $\Delta$ decreases the plot becomes rather continuous and resembles more closely the one in Fig.~\ref{fig:variational solutions}.}
	\label{fig:actual variational solutions}
\end{figure}

\subsection{The self-consistent method}

This method allows for a crystalline interpretation of $\Delta$. First, define the \textbf{un-pairing operator}
\[
	\hat{\phi}_\mathbf{k} \equiv  \hat{c}_{-\mathbf{k}\downarrow} \hat{c}_{\mathbf{k}\uparrow}
\]
that un-pairs a pair with opposite momenta and spins. The order of operator in the definition is important. It lets us rewrite the hamiltonian as
\[
	\hat{\bm H} - \mu\hat{N} = \sum_{\mathbf{k}} \xi_{\mathbf{k}} \hat n_\mathbf{k} + \sum_{\mathbf{k}\mathbf{k}'} V_{\mathbf{k}-\mathbf{k}'} \hat{\phi}_\mathbf{k}^\dagger \hat{\phi}_{\mathbf{k}'}
\]
with $\hat{n}_\mathbf{k} \equiv \hat{c}_{\mathbf{k}\uparrow}^\dagger \hat{c}_{\mathbf{k}\uparrow} + \hat{c}_{\mathbf{k}\downarrow}^\dagger \hat{c}_{\mathbf{k}\downarrow}$. In order to get this expression fermionic commutation rules have been used. We can substitute the pairing operator by its fluctuation around the mean value,
\[
	\hat{\phi} = \ev{\hat{\phi}} + \delta\hat{\phi}
\]
Then, substituting in the potential term and neglecting quadratic contributions,
\[
	V_{\mathbf{k}-\mathbf{k}'} \hat{\phi}_\mathbf{k}^\dagger \hat{\phi}_{\mathbf{k}'} = V_{\mathbf{k}-\mathbf{k}'}  \big\langle\hat{\phi}_\mathbf{k}^\dagger\big\rangle \ev{\hat{\phi}_{\mathbf{k}'}}
	+ V_{\mathbf{k}-\mathbf{k}'} \delta\hat{\phi}_\mathbf{k}^\dagger \ev{\hat{\phi}_{\mathbf{k}'}}
	+ V_{\mathbf{k}-\mathbf{k}'} \big\langle\hat{\phi}_\mathbf{k}^\dagger\big\rangle \delta\hat{\phi}_{\mathbf{k}'}
	+ \cdots
\]
The next step is quite of a turnaround: substituting only in the linear terms the same expression, $\delta\hat{\phi} = \hat{\phi}-\ev{\hat{\phi}}$, we get
\[
	V_{\mathbf{k}-\mathbf{k}'} \hat{\phi}_\mathbf{k}^\dagger \hat{\phi}_{\mathbf{k}'} = - V_{\mathbf{k}-\mathbf{k}'}  \big\langle\hat{\phi}_\mathbf{k}^\dagger\big\rangle \ev{\hat{\phi}_{\mathbf{k}'}}
	+ V_{\mathbf{k}-\mathbf{k}'} \hat{\phi}_\mathbf{k}^\dagger \ev{\hat{\phi}_{\mathbf{k}'}}
	+ V_{\mathbf{k}-\mathbf{k}'} \big\langle\hat{\phi}_\mathbf{k}^\dagger\big\rangle\hat{\phi}_{\mathbf{k}'}
	+ \cdots
\]
This kind of argument may seem circular, as it is, and it only holds if higher-than-linear terms are in effect negligible. This kind of approach falls under \textbf{mean-field theory}. Apart from constant shift to the energy, we get
\[
	\hat{\bm H} - \mu\hat{N} \simeq (\mathrm{constant}) + \sum_{\mathbf{k}} \xi_{\mathbf{k}} \hat n_\mathbf{k} + 
	\sum_{\mathbf{k}\mathbf{k}'} V_{\mathbf{k}-\mathbf{k}'} \hat{\phi}_\mathbf{k}^\dagger \ev{\hat{\phi}_{\mathbf{k}'}} + \sum_{\mathbf{k}\mathbf{k}'} V_{\mathbf{k}-\mathbf{k}'}  \big\langle\hat{\phi}_\mathbf{k}^\dagger \big\rangle \hat{\phi}_{\mathbf{k}'}
\]
Now we define
\[
	\Delta_\mathbf{k} \equiv \sum_{\mathbf{k}'} V_{\mathbf{k}-\mathbf{k}'} \ev{\hat{\phi}_{\mathbf{k}'}}
	\quad\implies\quad
	\Delta_\mathbf{k}^* = \sum_{\mathbf{k}'} V_{\mathbf{k}-\mathbf{k}'}^* \big\langle\hat{\phi}_{\mathbf{k}'}^\dagger\big\rangle
\]
Note that $V_{\mathbf{k}-\mathbf{k}'}^* = V_{\mathbf{k}'-\mathbf{k}}$. Thanks to this we have
\[
	\hat{\bm H} - \mu\hat{N} \simeq (\mathrm{constant}) + \sum_{\mathbf{k}} \xi_{\mathbf{k}} \hat n_\mathbf{k} + 
	\sum_\mathbf{k} \Delta_\mathbf{k} \hat{\phi}_\mathbf{k}^\dagger + \sum_\mathbf{k} \Delta_\mathbf{k}^* \hat{\phi}_\mathbf{k}
\]
where in the last term relabeling $\mathbf{k}' \to \mathbf{k}$ has been used. We now define the shifted hamiltonian $\hat{\mathcal{H}} \equiv \hat{\bm H} - \mu\hat{N} - (\mathrm{constant})$. Getting the equation compact,
\[
	\hat{\mathcal{H}} = \sum_\mathbf{k} \lrS{\xi_\mathbf{k} \hat{c}_{\mathbf{k}\uparrow}^\dagger \hat{c}_{\mathbf{k}\uparrow} + \xi_\mathbf{k} \hat{c}_{\mathbf{k}\downarrow}^\dagger \hat{c}_{\mathbf{k}\downarrow} + \Delta_\mathbf{k} \hat{c}_{\mathbf{k}\uparrow}^\dagger \hat{c}_{-\mathbf{k}\downarrow}^\dagger + \Delta_\mathbf{k}^*	 \hat{c}_{-\mathbf{k}\downarrow} \hat{c}_{\mathbf{k}\uparrow} }
\]
We now use some fermionic commutation relations and the symmetry of the dispersion relation $\xi_\mathbf{k} = \xi_{-\mathbf{k}}$,
\[
\begin{aligned}
	\hat{\mathcal{H}} &= \sum_\mathbf{k} \lrS{\xi_\mathbf{k} \hat{c}_{\mathbf{k}\uparrow}^\dagger \hat{c}_{\mathbf{k}\uparrow} - \xi_\mathbf{k}
	\hat{c}_{-\mathbf{k}\downarrow} \hat{c}_{-\mathbf{k}\downarrow}^\dagger + \Delta_\mathbf{k} \hat{c}_{\mathbf{k}\uparrow}^\dagger \hat{c}_{-\mathbf{k}\downarrow}^\dagger + \Delta_\mathbf{k}^* \hat{c}_{\mathbf{k}\uparrow} \hat{c}_{-\mathbf{k}\downarrow} } \\
	&= \sum_\mathbf{k} \begin{bmatrix}
		\hat{c}_{\mathbf{k}\uparrow}^\dagger & \hat{c}_{-\mathbf{k}\downarrow}
	\end{bmatrix} \begin{bmatrix}
		\xi_\mathbf{k} & \Delta_\mathbf{k} \\
		\Delta_\mathbf{k}^* & - \xi_\mathbf{k}
	\end{bmatrix} \begin{bmatrix}
		\hat{c}_{\mathbf{k}\uparrow} \\
		\hat{c}_{-\mathbf{k}\downarrow}^\dagger
	\end{bmatrix} \\
	&= \sum_\mathbf{k} \hat{\Phi}_\mathbf{k}^\dagger D_\mathbf{k} \hat{\Phi}_\mathbf{k}
\end{aligned}
\]
with $\hat{\Phi}_\mathbf{k}$ the spinorial operator in vector form and $D_\mathbf{k}$ the central matrix. It is easy to see that
\[
	D_\mathbf{k} = \Re{\Delta_\mathbf{k}} \sigma^1 + \Im{\Delta_\mathbf{k}} \sigma^2 + \xi_\mathbf{k} \sigma^3
\]
with $\sigma^i$ the $i$-th Pauli matrix. Thus this hamiltonian is the one for a spin in a (pseudo)magnetic field $\mathbf{b}_\mathbf{k}$ given by
\[
	\mathbf{b}_\mathbf{k} = \begin{bmatrix}
		\Re{\Delta_\mathbf{k}} \\ \Im{\Delta_\mathbf{k}} \\ \xi_\mathbf{k}
	\end{bmatrix}
\]
Analogously we define the (pseudo)spin components as
\[
	\lrS{\hat{\bm{\sigma}}_\mathbf{k}}^i \equiv \hat{\Phi}_\mathbf{k}^\dagger \sigma^i \hat{\Phi}_\mathbf{k}
\]
and the hamiltonian is reduced to the simple form
\[
	\hat{\mathcal{H}} \equiv \sum_\mathbb{k} \mathbf{b}_\mathbf{k} \cdot \hat{\bm{\sigma}}_\mathbf{k}
\]
This is the problem of a spin of magnitude $1$ in a field.
The eigenvalues are well-known to be plus or minus the intensity of the field,
\[
	\pm \lambda_\mathbf{k} = \pm \sqrt{\xi_\mathbf{k}^2 + \abs{\Delta_\mathbf{k}}^2}
\]
as can be seen easily starting from $D_\mathbf{k}$.
Now $\lambda_\mathbf{k}^\pm$ represent the spectrum of the system. We will dedicate the next section to comment the result. First, indicating by $U_\mathbf{k}$ the matrix that diagonalizes $D_\mathbf{k}$,
\[
	\Lambda_\mathbf{k} \equiv
	U_\mathbf{k} D_\mathbf{k} U_\mathbf{k}^\dagger = \begin{bmatrix}
		\lambda_\mathbf{k} & \\
		& -\lambda_\mathbf{k}
	\end{bmatrix}
\]
it is clear that
\[
	\hat{\Phi}_\mathbf{k}^\dagger D_\mathbf{k} \hat{\Phi}_\mathbf{k} = \hat{\Phi}_\mathbf{k}^\dagger U_\mathbf{k}^\dagger U_\mathbf{k} D_\mathbf{k} U_\mathbf{k}^\dagger U_\mathbf{k} \hat{\Phi}_\mathbf{k} = \hat{\Gamma}_\mathbf{k}^\dagger \Lambda_\mathbf{k} \hat{\Gamma}_\mathbf{k}
\]
where we defined the spinor in the eigenvectors basis,
\[
	\hat{\Gamma}_\mathbf{k} \equiv U_\mathbf{k} \hat{\Phi}_\mathbf{k}
\]
Now: the matrix $U_\mathbf{k}$ will surely mix up the operators $\hat{c}_{\mathbf{k}\uparrow}$ and $\hat{c}_{-\mathbf{k}\downarrow}^\dagger$ inside $\hat{\Phi}_\mathbf{k}$. The essence of the Bogoliubov approach to quadratic hamiltonian is, in fact, to find an optimal linear combination of second quantization operators (and fields) that reduces the whole problem to a system of new free fermions, born by some kind of combination of ``physical'' particles. This will become clear in a moment.

\begin{figure}
	\centering
	\def\DeltaParameter{0.6}
\def\XiParameter{0.4}
\begin{tikzpicture}
	\begin{axis}[
		axis x line=center,
		axis y line=center,
		axis z line=center,
		axis on top,
		xlabel={$x$},
		ylabel={$y$},
		zlabel={$z$},
		xlabel style={left},
		ylabel style={below},
		zlabel style={above},
		xtick={-\DeltaParameter,\DeltaParameter},
		ytick=\empty,
		ztick={-\XiParameter,\XiParameter},
		xticklabels={$-\Delta_\mathbf{k}$,$\Delta_\mathbf{k}$},
		yticklabel=\empty,
		zticklabels={$-\xi_\mathbf{k}$,$\xi_\mathbf{k}$},
		xticklabel style=\empty,
		yticklabel style=\empty,
		zticklabel style={right},
		xmin=-1, xmax=1,
		ymin=-0.25, ymax=0.25,
		zmin=-1, zmax=1,
		view/h=193,
		view/v=15,
		scale=1.5]
		
		% Dashed lines
		\draw[color=gray!70,dashed]
			(axis cs:\DeltaParameter,0,0) -- (axis cs:\DeltaParameter,0,\XiParameter);
		\draw[color=gray!70,dashed]
			(axis cs:0,0,\XiParameter) -- (axis cs:\DeltaParameter,0,\XiParameter);
		\draw[color=gray!70,dashed]
			(axis cs:-\DeltaParameter,0,0) -- (axis cs:-\DeltaParameter,0,-\XiParameter);
		\draw[color=gray!70,dashed]
			(axis cs:0,0,-\XiParameter) -- (axis cs:-\DeltaParameter,0,-\XiParameter);
		
		% Eigenstates
		\filldraw[color=lev] 
			(axis cs:\DeltaParameter,0,\XiParameter) circle (1.2pt) node[anchor=south east]
				{$-\lambda_\mathbf{k}$};
		\filldraw[color=red] 
			(axis cs:-\DeltaParameter,0,-\XiParameter) circle (1.2pt) node[anchor=north west]
				{$\lambda_\mathbf{k}$};
		
		% Pseudo field
		\draw[color=lev,-stealth]
			(axis cs:0,0,0) -- (axis cs:\DeltaParameter,0,\XiParameter) node[anchor=north east,midway]
				{$\mathbf{b}_\mathbf{k}$};
		
		% Angle
		\begin{scope}[canvas is zx plane at y=0]
			\draw[color=lev] 
				(\XiParameter/2,0) arc [start angle=0,end angle=atan(\DeltaParameter/\XiParameter),radius=\XiParameter/2] node[anchor=south,midway]
					{$2\theta_\mathbf{k}$};
		\end{scope}
		
	\end{axis}
\end{tikzpicture}
	\caption{Representation of the (pseudo)magnetic field $\mathbf{b}_\mathbf{k}$. Obviously the notation $xyz$ is unphysical and in any way related to real space. The dots represent the eigenvectors of the problem, and the relative eigenvalue is indicated. As in any problem of this kind, the lowest eigenvalue $-\lambda_\mathbf{k}$ is reached for a combination of spinors perfectly aligned with the field, while the highest eigenvalue is its antipodal point.}
	\label{fig:pseudo magnetic field}
\end{figure}

So, we need $U_\mathbf{k}$. Since this problem is physically equivalent to finding the eigenstates of a spin in tilted magnetic field, we already know that the diagonal form of the matrix is obtained by applying a rotation that aligns the $z$ axis with the field. To write the rotation, we use the result of Sec.~\ref{subsec:the variational method}, which embeds a real $\Delta_\mathbf{k}$. Then the (pseudo)magnetic field is rotated on the $zx$ plane by an angle $2\theta_\mathbf{k}$ with respect to the $z$ axis, such that
\[
	\frac{\Delta_\mathbf{k}}{\xi_\mathbf{k}} =  \tan 2\theta_\mathbf{k}
\]
and we have already seen this relation, in Eq.~\eqref{eq:delta variational definition}! In Fig.~\ref{fig:pseudo magnetic field} a sketch of the field and its eigenvectors is reported. To align the $z$ axis with the field we need to rotate the $zx$ plane by an angle $2\theta_\mathbf{k}$ clockwise; this is equivalent to rotating the vectors by the same angle anti-clockwise.
The $\mathrm{SO}(3)$ representation of this rotation is
\[
	U_\mathbf{k}^{\mathrm{SO}(3)} = \begin{bmatrix}
		\cos 2\theta_\mathbf{k} & 0 & \sin 2\theta_\mathbf{k} \\
		0 & 1 & 0 \\
		- \sin 2\theta_\mathbf{k} & 0 & \cos 2\theta_\mathbf{k}
	\end{bmatrix}
\]
We need its mapping onto its $\mathrm{SU}(2)$ version. In general, a rotation of angle $\alpha$ around the versor $\versor{n}$ is represented in the group by
\[
	\exp{- i \frac{\alpha}{2} \versor{n} \cdot \bm{\sigma}}
\]
and for us $\versor{n} = \versor{y}$, $\alpha = -2\theta_\mathbf{k}$; expanding:
\[
\begin{aligned}
	U_\mathbf{k}^{\mathrm{SU}(2)} &= \mathbb{1} \cos \theta_\mathbf{k} + i \versor{n} \cdot \bm{\sigma} \sin \theta_\mathbf{k}\\
	&= \mathbb{1} \cos \theta_\mathbf{k} + i \sigma^2 \sin \theta_\mathbf{k} = \begin{bmatrix}
		\cos \theta_\mathbf{k} & -\sin \theta_\mathbf{k} \\
		\sin \theta_\mathbf{k} & \cos \theta_\mathbf{k}
	\end{bmatrix}
\end{aligned}
\]
We omit now the representation superscript. It can be checked easily that
\[
	\begin{bmatrix}
		\cos \theta_\mathbf{k} & \sin \theta_\mathbf{k} \\
		- \sin \theta_\mathbf{k} & \cos \theta_\mathbf{k}
	\end{bmatrix} 
	\begin{bmatrix}
		\xi_\mathbf{k} & \Delta_\mathbf{k} \\
		\Delta_\mathbf{k}^* & - \xi_\mathbf{k}
	\end{bmatrix}
	\begin{bmatrix}
		\cos \theta_\mathbf{k} & -\sin \theta_\mathbf{k} \\
		\sin \theta_\mathbf{k} & \cos \theta_\mathbf{k}
	\end{bmatrix} =
	\begin{bmatrix}
		\lambda_\mathbf{k} & \\
		& -\lambda_\mathbf{k}
	\end{bmatrix}
\]

Then we can read the spinor in the eigenvectors basis just by applying the rotation,
\[
\begin{aligned}
	\hat{\Gamma}_\mathbf{k} &= U_\mathbf{k} \hat{\Phi}_\mathbf{k} \\
	&= \begin{bmatrix}
		\cos \theta_\mathbf{k} & -\sin \theta_\mathbf{k} \\
		\sin \theta_\mathbf{k} & \cos \theta_\mathbf{k}
	\end{bmatrix} \begin{bmatrix}
		\hat{c}_{\mathbf{k}\uparrow} \\
		\hat{c}_{-\mathbf{k}\downarrow}^\dagger
	\end{bmatrix} \\
	&= \begin{bmatrix}
		\cos \theta_\mathbf{k} \hat{c}_{\mathbf{k}\uparrow} -\sin \theta_\mathbf{k} \hat{c}_{-\mathbf{k}\downarrow}^\dagger \\
		\sin \theta_\mathbf{k} \hat{c}_{\mathbf{k}\uparrow} + \cos \theta_\mathbf{k} \hat{c}_{-\mathbf{k}\downarrow}^\dagger
	\end{bmatrix} \equiv \begin{bmatrix}
		\hat{\gamma}_{\mathbf{k}\uparrow} \\
		\hat{\gamma}_{-\mathbf{k}\downarrow}^\dagger
	\end{bmatrix}
\end{aligned}
\]
where in the last step we defined a pair of new fermionic operators, $\hat{\gamma}_{\mathbf{k}\uparrow}$ and $\hat{\gamma}_{-\mathbf{k}\downarrow}^\dagger$. It makes sense to define them this way, because in
\[
	\hat{\gamma}_{\mathbf{k}\uparrow} \equiv \cos \theta_\mathbf{k} \hat{c}_{\mathbf{k}\uparrow} -\sin \theta_\mathbf{k} \hat{c}_{-\mathbf{k}\downarrow}^\dagger
\]
to create an excitation with momentum $-\mathbf{k}$ and spin $\downarrow$ is kind of equivalent to annihilating an excitation with momentum $\mathbf{k}$ and spin $\uparrow$. The two operations are not equivalent with respect to the number of electrons in the system, however the change in total momentum and spin is the same. An analogous consideration holds for
\[
	\hat{\gamma}_{-\mathbf{k}\downarrow}^\dagger \equiv \sin \theta_\mathbf{k} \hat{c}_{\mathbf{k}\uparrow} + \cos \theta_\mathbf{k} \hat{c}_{-\mathbf{k}\downarrow}^\dagger
\]
For what concerns the $\hat{\gamma}$ operators, the subscripts must be intended in this way and do not have the physical meaning they have for the $\hat{c}$ operators. It can be checked that the $\hat{\gamma}$ operators obey the common anti-commutation rules. Thanks to these transformations the hamiltonian reads
\[
	\hat{\mathcal{H}} = \sum_\mathbf{k} \hat{\Gamma}_\mathbf{k}^\dagger \begin{bmatrix}
		\lambda_\mathbf{k} & \\
		& -\lambda_\mathbf{k}
	\end{bmatrix} \hat{\Gamma}_\mathbf{k} =
	\sum_\mathbf{k} \lrS{\lambda_\mathbf{k} \hat{\gamma}_{\mathbf{k}\uparrow}^\dagger \hat{\gamma}_{\mathbf{k}\uparrow} - \lambda_\mathbf{k} \hat{\gamma}_{-\mathbf{k}\downarrow}\hat{\gamma}_{-\mathbf{k}\downarrow}^\dagger}
\]
and, using fermionic rules and the dispersion symmetry $\lambda_\mathbf{k} = \lambda_{-\mathbf{k}}$, we finally have
\[
	\hat{\mathcal{H}} = \sum_\mathbf{k} \lambda_\mathbf{k} \lrS{ \hat{\gamma}_{\mathbf{k}\uparrow}^\dagger \hat{\gamma}_{\mathbf{k}\uparrow} + \hat{\gamma}_{\mathbf{k}\downarrow}^\dagger\hat{\gamma}_{\mathbf{k}\downarrow}}
\]
Now everything is clear. To use the solemn and inspiring words of one of the major art pieces of the author's country, ``Nessuno è più basito, chiaro? Nessuno è più basito, nessuno è sorpreso, ognuno di voi ha capito tutto. Nei primi piani fate sì con la testa, che avete capito e state sereni''. As long as the un-pairing operator fluctuates negligibly, the BCS hamiltonian can be mapped on a system of free fermions described by the $\hat{\gamma}$ operators. Those fermions are divided in two classes, $\uparrow$ and $\downarrow$, distinguished by the change in total spin the system obtains when one of these fermions is added or removed. Both classes have dispersion
\[
	\lambda_\mathbf{k} = \sqrt{\xi_\mathbf{k}^2 + \Delta_\mathbf{k}^2}
\]

Note that the orientation in abstract space of the (pseudo)field obviously depends on $\mathbf{k}$. Take Fig.~\ref{fig:pseudo magnetic field}: if we use the known value
\[
	\Delta_\mathbf{k} = \Delta A\lrR{\mathbf{k}}
	\quad\qq{with}\quad
	\Delta = 2 \hbar\omega_D e^{-2/\rho_0 V_0}
\]
inside the interaction shell $\Delta_\mathbf{k} = \Delta$, while the $\xi_\mathbf{k}$ component will increase in magnitude as move $\mathbf{k}$ away from the Fermi surface, in both directions. Eventually we exit the shell: for $\abs{\xi_\mathbf{k}} > \hbar\omega_D$, we have $\Delta_\mathbf{k} = 0$ and therefore the magnetic field has only the $z$ component and $\theta_\mathbf{k}=0$.  This means, as it is evident from the beginning, that the Bogoliubov $\hat{\gamma}$ operators coincide with the electron $\hat{c}$ operators, and the hamiltonian can be written as
\[
	\hat{\mathcal{H}} = \sum_{\abs{\xi_\mathbf{k}} > \hbar\omega_D} \xi_\mathbf{k} \lrS{ \hat{c}_{\mathbf{k}\uparrow}^\dagger \hat{c}_{\mathbf{k}\uparrow} + \hat{c}_{\mathbf{k}\downarrow}^\dagger \hat{c}_{\mathbf{k}\downarrow}} + \sum_{\abs{\xi_\mathbf{k}} < \hbar\omega_D} \sqrt{\xi_\mathbf{k}^2 + \Delta^2} \lrS{ \hat{\gamma}_{\mathbf{k}\uparrow}^\dagger \hat{\gamma}_{\mathbf{k}\uparrow} + \hat{\gamma}_{\mathbf{k}\downarrow}^\dagger \hat{\gamma}_{\mathbf{k}\downarrow}}
\]
It is time, now, to pass to the announced comment about the meaning of $\Delta$. All we said so far works for $T=0$, of course.

\section{The Importance of Being Gapped \newline \small (a Trivial Comedy for Superconducting People)}

\begin{figure}
	\centering
	\def\ShellWidth{0.72}
\def\BindingEnergy{0.36}
\begin{tikzpicture}
	\begin{axis}[
		axis x line=center,
		axis y line=center,
		axis equal image,
		xlabel={$\xi_\mathbf{k}$},
		ylabel={$\lambda_\mathbf{k}$},
		xlabel style=below,
		ylabel style=above,
		xtick={-\ShellWidth,\ShellWidth},
		ytick=\empty,
		xticklabels={$-\hbar\omega_D\quad$,$\quad\hbar\omega_D$},
		yticklabel=\empty,
		xticklabel style=\empty,
		yticklabel style=\empty,
		xmin=-1.1, xmax=1.1,
		ymin=-0.1, ymax=1.1
		]
		
		% Interaction region
		\fill[color=lev!30,opacity=0.5] 
			(axis cs:-\ShellWidth,0) rectangle (axis cs:\ShellWidth,1);
		
		% Dispersion
		\addplot[color=lev,domain=-1:-\ShellWidth] 
			{abs(x)};
		\addplot[color=lev,dashed,domain=-\ShellWidth:\ShellWidth] 
			{abs(x)};
		\addplot[color=lev,domain=\ShellWidth:1] 
			{abs(x)};
		\addplot[color=lev,smooth,domain=-\ShellWidth:\ShellWidth] 
			{sqrt(x^2+\BindingEnergy^2)};
			
		% Gap
		\draw[stealth-stealth]
			(axis cs:0.03,0) -- (axis cs:0.03,\BindingEnergy) node[midway,anchor=south west]
				{$\Delta$};
	\end{axis}
\end{tikzpicture}
	\caption{Sketch of the excitation spectrum for the superconductor (solid line) compared to the spectrum of a normal metal (dashed line).}
\label{fig:excitation spectrum}
\end{figure}

We start by taking a normal metal at zero temperature, described by its filled Fermi sphere. If we analyze its spectrum, its first excited states are electron states with energy $\xi_\mathbf{k} > 0$ outside the Fermi sphere and hole states with energy $-\xi_\mathbf{k} > 0$ inside the Fermi sphere (holes dispersion has opposite sign with respect to the electronic dispersion). Then the excitation spectrum is given by $\abs{\xi_\mathbf{k}}$, at least immediately around the Fermi surface. In Fig.~\ref{fig:excitation spectrum} it is represented by the dashed line inside the interaction region, and the solid line outside. The \textit{very} important thing to notice is that, for any amount of (small) energy we pump into the system, for the normal metal two excited configurations capable of absorbing such energy exist. This, of course, is a consequence of the fact that in a metal the band are crossed by the Fermi level and so the dispersion is \textit{gapless}.

Now, in Fig.~\ref{fig:excitation spectrum} the solid line represents the excitation spectrum around the Fermi surface. Apart from the discontinuities at the region boundaries, a mere consequence of the approximations we did on the interaction potential, deep in the shell the excitation of spectrum is \textit{gapped}. This means that for vanishing excitation energy [...]