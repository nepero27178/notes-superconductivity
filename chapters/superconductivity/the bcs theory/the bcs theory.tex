\chapter{The BCS theory}\chaptertoc{}\label{chap:the bcs theory}

It is time to develop formally and completely the theory by Bardeen, Cooper and Schrieffer. We saw in the last chapter how electrons form Cooper pairs, because apparently in superconductors charges flow coupled. The question now is: why is the coupling of electrons (a rather weak one, also) necessary for the exotic phenomena of superconductivity, like resistanceless flow of charge and Meissner effect?

\section{The BCS hamiltonian}

From last chapter we know that the phonon-mediated effective hamiltonian is given by
\[
	\hat{\bm H} = \sum_{\mathbf{k}} \epsilon_{\mathbf{k}} \lrR{\hat{c}_{\mathbf{k}\uparrow}^\dagger \hat{c}_{\mathbf{k}\uparrow} + \hat{c}_{\mathbf{k}\downarrow}^\dagger \hat{c}_{\mathbf{k}\downarrow}} + \sum_{\mathbf{k}\mathbf{k}'} V_{\mathbf{k}-\mathbf{k}'} \lrR{\hat{c}_{\mathbf{k}\uparrow}^\dagger \hat{c}_{\mathbf{k}'\uparrow}} \lrR{\hat{c}_{-\mathbf{k}\downarrow}^\dagger \hat{c}_{-\mathbf{k}'\downarrow}}
\]

\section{The Importance of Being Gapped,\\ \small a Trivial Comedy for Superconducting People}