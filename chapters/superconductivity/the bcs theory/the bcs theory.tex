\chapter{The BCS theory}\chaptertoc{}\label{chap: the bcs theory}

In the last chapters, we treated superconductivity on a purely macroscopical level. Chap.~\ref{chap:superconductivity and thermodynamics} made use of classical electrodynamics, while Chaps.~\ref{chap:ginzburg-landau theory of superconductivity} and \ref{chap:conventional superconductors} of Ginzburg-Landau theory; within this last approach, a lot of striking features of superconductivity have been derived, with much focus on phenomenology. It is time to pass to a much shorter length scale, investigating the profound causes of superconductivity in the behavior of the quantum particles involved. The theory describing such regime is the one elaborated by Bardeen, Cooper and Schrieffer (BCS), and published in their historical article \citetitle{PhysRev.108.1175} \cite{PhysRev.108.1175} in \citeyear{PhysRev.108.1175}.

Briefly, the BCS theory accurately explains superconductivity as the phase of matter in which the dominant interaction acting on a system of electrons inside a material is an effective interaction arising from the crystalline structure of the material itself: couples of electrons experience an \textbf{attractive interaction}, mediated by a phonon. The lattice vibrations act as carriers of energy and momentum and make it possible for electron pairs to form bound states. Such bosonic couples, called \textbf{Cooper pairs}, enter the superconducting state by forming a Bose-Einsten condensate. We will delve into these concepts in this chapter.

\begin{cit}{grosso2000solid}{18}
	The history of superconductivity is full of fascinating surprises and challenging developments. The milestone work of Kamerlingh Onnes in 1911 on the electrical resistivity of mercury has opened a new world to the physical investigation, and the discovery of high-$T_c$ superconductivity in barium-doped lanthanum cuprate, by Bednorz and Müller in 1986, has given a novel impetus to the subject. 
	
	The microscopic origin of superconductivity is linked to the possible occurrence of a (small) effective attractive interaction between conduction electrons (or valence holes in p-type conductors) and the consequent formation of electron pairs (or hole pairs), at sufficiently low temperatures. The mechanism of electron pairing is at the origin of perfect conductivity, perfect diamagnetism, anomalous specific heat and thermodynamical properties, magnetic flux quantization, coherent tunneling, and several other effects in superconductors. Empirical laws and semi-empirical models have accompanied the accumulation of the wide and rich phenomenology of superconductors. Eventually, the fundamental work of Bardeen, Cooper, and Schrieffer (1957) has transformed an endless list of peculiar effects and conjectures into a logically consistent theoretical framework. [...] without its concepts no serious discussion would be possible at all.
\end{cit}

\section{What if electrons attract?}

Many experiments exist, showing that the elementary ``object'' inside a superconductor has charge $q=2e$. This can be verified, for example, measuring the quantization of the magnetic flux inside a superconducting sample. The flux gets quantized as
\[
	\Phi = n\frac{h}{\abs{q}} = n \frac{h}{2\abs{e}} 
\]
This general rule, that seems to be obeyed flawlessly in the superconducting phase, indicates that such object is \textbf{a pair of electrons}.

Moreover, the superconducting transition exhibits many similarities with the superfluid transition of liquid Helium, which is well known to be a Bose-Einstein condensation process. As it turns out, a superconductor is a condensate state. To produce a condensate, then, we need bosons. Electron pairs, seen as composite objects, are bosons.

Other arguments point in the same direction: superconductivity is the condensation of a system of electrons pairs. This is the corner stone of the BCS theory. To make a pair, we need an \textbf{attractive interaction} between electrons: we know they interact via the (screened) Coulomb interaction and the Pauli principle, so it may seem strange to look for some kind of attraction; we assume they somehow attract, and see if they form bound states -- which are, pairs.

\subsection{Bound states}

Consider two interacting electrons in $D$ dimensions, with hamiltonian
\[
	\hat H = \frac{\hat{\mathbf{p}}_1^2}{2m} + \frac{\hat{\mathbf{p}}_2^2}{2m} + V\lrR{\hat{\mathbf{x}}_1-\hat{\mathbf{x}}_2}
\]
with obvious notation. The same hamiltonian can be decomposed in the sum of the center of mass part and the relative part,
\[
	\hat H = \lrS{\frac{\hat{\mathbf{P}}^2}{2M}} + \lrS{\frac{\hat{\mathbf{p}}^2}{2\mu} + V \lrR{\hat{\mathbf{x}}}}
\]
with
\[
	\mathbf{P} \equiv \mathbf{p}_1 + \mathbf{p}_2
	\qquad
	\mathbf{X} \equiv \frac{\mathbf{x}_1 + \mathbf{x}_2}{2}
	\qquad
	\mathbf{p} \equiv \frac{\mathbf{p}_1 - \mathbf{p}_2}{2}
	\qquad
	\mathbf{x} \equiv \mathbf{x}_1 - \mathbf{x}_2
\]
and
\[
	M = 2m
	\qquad
	\mu = \frac{m}{2}
\]
Assuming overall translational symmetry, the wavefunction can be factorized as
\[
	\psi(\mathbf{x}_1,\mathbf{x}_2) = \Phi(\mathbf{X}) \phi(\mathbf{x})
\]
where $\Phi$ is the wavefunction of the center of mass, and $\phi$ is the relative wavefunction.

Now, consider a local interaction, on a ``small'' length scale. We may start by considering the perfectly local contact-attractive interaction,
\[
	V \lrR{\mathbf{x}} \equiv - V_0 \delta^{(D)} \lrR{\mathbf{x}}
	\quad\qq{with}\quad
	V_0 > 0
\]
Here we are neglecting the Coulomb interaction. It is reasonable to do so if such interaction is screened, as it commonly is in materials. For the Coulomb interaction to be screened we need the whole electron liquid background: for more details on this subject, check the vast book \citetitle{giuliani2008quantum} \cite{giuliani2008quantum} by \citeauthor{giuliani2008quantum}. Let us forget for a moment both the electron liquid and the Coulomb interaction, and proceed with two locally interacting chargeless fermions. This evidently incoherent argument is necessary to highlight, in the following, the essential collective nature of the attractive interaction.

The Schrödinger's Equation for the relative part of the wavefunction is given by
\[
	\lrS{\frac{\hat{\mathbf{p}}^2}{2\mu} + V \lrR{\hat{\mathbf{x}}}} \phi(\mathbf{x}) = - E^{(\mathrm{b})} \phi(\mathbf{x})
\]
where the eigenvalue $- E^{(\mathrm{b})} < 0$ indicates the binding energy. Consider now the complete basis of orthonormal plane waves,
\[
	w_\mathbf{k} \lrR{\mathbf{x}} = L^{-D/2} e^{i \mathbf{k} \cdot \mathbf{x}}
\]
with $L^D$ the total volume. The wavefunction can be decomposed as
\[
	\phi\lrR{\mathbf{x}} = \sum_\mathbf{k} \alpha_\mathbf{k} w_\mathbf{k} \lrR{\mathbf{x}}
\]
Thus, inserting the above decomposition in the Schrödinger's Equation and projecting onto the plane wave $w_\mathbf{k} \lrR{\mathbf{x}}$, we obtain
\[
	\epsilon_\mathbf{k} \alpha_\mathbf{k} + \sum_{\mathbf{k}'} V_{\mathbf{k}-\mathbf{k}'} \alpha_{\mathbf{k}'} = - E^{(\mathrm{b})} \alpha_\mathbf{k}
	\quad\qq{with}\quad
	\epsilon_\mathbf{k} = \frac{\hbar^2 \abs{\mathbf{k}^2}}{2\mu}
\]
and where the Fourier transform of the interaction potential is intended,
\[
	V_{\mathbf{k}-\mathbf{k}'} = \frac{1}{L^D} \int_{\R^D} d\mathbf{x} \, V \lrR{\mathbf{x}} e^{i(\mathbf{k}-\mathbf{k}') \cdot \mathbf{x}} = - V_0
\]
since the potential is \textit{delta-like}.  Then,
\[
	\lrR{\epsilon_\mathbf{k} + E^{(\mathrm{b})}} \alpha_\mathbf{k} = V_0 \sum_{\mathbf{k}'} \alpha_{\mathbf{k}'}
\]
It follows:
\[
	\alpha_\mathbf{k} = \frac{V_0}{\epsilon_\mathbf{k} + E^{(\mathrm{b})}} \sum_{\mathbf{k}'} \alpha_{\mathbf{k}'}
\]
then, summing over $\mathbf{k}$, the coefficient $\sum \alpha$ can be simplified both sides, leaving the self-consistency equation
\[
	\sum_\mathbf{k} \frac{V_0}{\epsilon_\mathbf{k} + E^{(\mathrm{b})}} = 1
\]
Assuming a large volume we can make approximate the momenta as continuous. A little caution is here needed: we approximate the potential as ``perfectly local'', which means that the length scale over which it drops to zero is much smaller than any physical length scale involved in the system. Having neglected the Coulomb long-range interaction, we understand the relevant length here cited is of the order of the particle dimension: our fermions are dimensionless points. Briefly, we should integrate on $\R^D \setminus s(2\pi/\Lambda)$ with $s(r)$ the sphere of radius $r$ in $D$ dimensions and $2\pi/\Lambda$ the said length for a properly defined momentum $\Lambda$.

This is equivalent to integrating over $\R^D$ a potential whose Fourier transform is constant for $\abs{\mathbf{k}} < \Lambda$ and (approximately and continuously) drops to zero for bigger momenta. Such potential is strongly localized, and \textit{delta-like} as seen ``from distant''. Defining $\kappa \equiv (2\pi)^D/ L^D V_0$ we have
\[
	\kappa = \int_{\abs{\mathbf{k}}<\Lambda} d^D \mathbf{k} \, \frac{1}{\epsilon_\mathbf{k} + E^{(\mathrm{b})}}
\]
The question is: at varying dimensionality $D$, is there a solution for any given $\kappa$?

\begin{enumerate}
	\item For $D=1$, the integral becomes
	\[
		\kappa = \int_{\abs{k}<\Lambda} dk \, \lrS{\frac{\hbar^2 k^2}{2\mu} + E^{(\mathrm{b})}}^{-1}
	\]
	The above function is solved by an infinite set of couples $(E^{(\mathrm{b})},\kappa)$; $\kappa$ is a continuous function of $E^{(\mathrm{b})}$. Moreover, for $E^{(\mathrm{b})} \to 0$ the integral presents an hyperbolic divergence, thus allowing for a $\kappa\to\infty$ solution. Then for any choice of $\kappa\in\R$ a solution exists. 
	
	The bound state is formed regardless of $\kappa$, which is, regardless of the attraction strength $V_0$.
	\item For $D=2$, we get
	\[
		\kappa = \int_{\abs{\mathbf{k}}<\Lambda} d^2 k \, \lrS{\frac{\hbar^2 k^2}{2\mu} + E^{(\mathrm{b})}}^{-1} = \pi \int_{k^2 < \Lambda^2} d k^2 \, \lrS{\frac{\hbar^2 k^2}{2\mu} + E^{(\mathrm{b})}}^{-1}
	\]
	where we used $d^2 k = 2\pi k dk = \pi dk^2$. The same argument of the point above holds: for $E^{(\mathrm{b})} \to 0$ the integral presents a logarithmic divergence, thus allowing for a $\kappa\to\infty$ solution. Then for any choice of $\kappa\in\R$ a solution exists.
	
	Also for $D=2$ the bound state is formed regardless of the attraction strength $V_0$.
	
	\item For $D>2$, we can use
	\[
		d^D \mathbf{k} = \Omega_D k^{D-1} dk
	\]
	with $\Omega_D$ the $D$-dimensional solid angle. Thus the integral becomes
	\[
		\kappa = \int_{\abs{\mathbf{k}}<\Lambda} d^D \mathbf{k} \, \lrS{\frac{\hbar^2 k^2}{2\mu} + E^{(\mathrm{b})}}^{-1} = \Omega_D \int_{k<\Lambda} dk \, k^{D-1} \lrS{\frac{\hbar^2 k^2}{2\mu} + E^{(\mathrm{b})}}^{-1} 
	\]
	Being $D-1 \ge 2$, this integral remains finite for any value of $E^{(\mathrm{b})}$, as long as the cutoff $\Lambda$ is finite. Moreover, the maximum value (which is finite and we denote by $\kappa^\star$) of the integral is recovered for $E^{(\mathrm{b})} \to 0$.
	
	For $D=3$ and in higher dimensions, two electrons form a bound state if $\kappa \le \kappa^\star$ -- or, if the interaction potential $V_0$ exceeds a certain threshold value.
\end{enumerate}

It looks like two chargeless fermions equipped with a local and attractive interaction cannot form a pair in three dimensions. This should limit the phenomenon of superconductivity to two-dimensional materials. Then, why do we have three-dimensional superconductors? 

\subsection{Bound states, considering statistics}\label{subsec:bound states, considering statistics}

We are missing something. As anticipated, to neglect the Coulomb interaction between electrons we need the whole electron liquid -- which is, we need a great number of electrons plus our two interacting electrons, interacting with all others only by the Pauli principle. This may seem a little modification; it is instead a huge one, because now this kind of attraction allows for bound states also for $D>2$. We may say that the electron pair is an object formed by two electrons directly and all others indirectly -- a collective configuration.

So, consider a system formed by the filled Fermi sphere plus two electrons, as described in the above paragraph. All the ``Fermi electrons'' prevent our two interacting electrons from occupying states inside the Fermi sphere. Another assumption can be made: in any way the attractive interaction arises, it is reasonable to assume that for electrons ``very distant'' from the Fermi surface the kinetic contribution is dominant and the effect of the attraction is negligible; this is equivalent to say that the maximum amount of energy the attraction can absorb for electrons of energy slightly bigger than $\epsilon_F$ is some amount $\delta \epsilon^\star$, and for $\epsilon_\mathbf{k} \gg \epsilon_F + \delta \epsilon^\star$ the potential drops to zero.

We assume that the shell is \textit{thin}, meaning $\delta\epsilon^\star \ll \epsilon_F$. Notice that to say that the potential has no components inside the Fermi sphere means that our pair cannot interact via the potential with the electrons inside, but only through the Pauli principle, thus being passively excluded from the sphere.
%To be completely correct we should extend the shell inside the sphere for a length $\delta\epsilon^\star$, in order for the potential to be able to extract 

\begin{figure}
	\centering
	\begin{tikzpicture}
	\filldraw[color=lev,fill=lev!60,fill opacity=0.5] (0,0) circle (7em);
	\fill[color=lev,fill=lev!30,fill opacity=0.5] (0,0) circle (8.5em);
	
	\fill[color=lev] (-25:7.4em) circle (1.2pt);
	\draw[color=lev,-stealth] (0,0)--(-25:7.4em);
	\node[color=lev,anchor=north east] at (-25:6em) {\small $\mathbf{k}$};	
	
	\fill[color=lev] (123:8.3em) circle (1.2pt);
	\draw[color=lev,-stealth] (0,0)--(123:8.3em);
	\node[color=lev,anchor=north east] at (123:6em) {\small $\mathbf{k}'$};
	
	\draw[color=lev!60,stealth-stealth] (0,0)--(72:7em);
	\node[color=lev!60,anchor=north west] at (72:4.25em) {\small $k_F$};	
	
	\draw[color=lev!60,stealth-stealth] (72:7em)--(72:8.5em);
	\node[color=lev!60,anchor=north west] at (72:8.1em) {\small $\delta k^\star$};	
\end{tikzpicture}
	\caption{Representation of the Fermi sphere, of radius $k_F$ in darker color, and the interaction shell of width $\delta k^\star$ in lighter color. The solid dots represents the two interacting electrons. The Fermi sphere is to be thought as filled with electrons, and interacting with the couple through Pauli exclusion principle.}
	\label{fig:fermi sphere and shell}
\end{figure}

We follow the same argument as the above section. 
The Schrödinger's Equation for the relative part of the wavefunction is
\[
	\lrS{\frac{\hat{\mathbf{p}}^2}{2\mu} + V \lrR{\hat{\mathbf{x}}}} \phi(\mathbf{x}) = \lrR{\epsilon_F - E^{(\mathrm{b})}} \phi(\mathbf{x})
\]
where now the eigenvalue is shifted by an amount $\epsilon_F$. In fact we consider a pair bound ``on top of the Fermi surface'', so we consider paired those states outside the Fermi sphere with energy lower than $\epsilon_F$. Proceeding with the plane wave expansion,
\[
	\epsilon_\mathbf{k} \alpha_\mathbf{k} + \sum_{\mathbf{k}'} V_{\mathbf{k}-\mathbf{k}'} \alpha_{\mathbf{k}'} = \lrR{\epsilon_F - E^{(\mathrm{b})}} \alpha_\mathbf{k}
\]
where now the potential has nonzero Fourier components only in the shell of width $\epsilon^\star$ around the Fermi sphere. Take Fig.~\ref{fig:fermi sphere and shell}: we approximate the potential as active only for those plane waves $\ket{\mathbf{k}}$ and $\ket{\mathbf{k}'}$ inside the shell of radius $k_F + \delta k^\star$ defined such that
\[
	\frac{\hbar^2}{2\mu} \lrR{k_F + \delta k^\star} = \epsilon_F + \delta \epsilon^\star
\]
We approximate the potential as
\[
	V_{\mathbf{k}-\mathbf{k}'} = - V_0 A\lrR{\mathbf{k}} A\lrR{\mathbf{k}'}
\]
where $A$ is the characteristic function of the shell,
\[
	A\lrR{\mathbf{k}} \equiv \theta\lrR{\abs{\mathbf{k}}-k_F}\theta\lrR{k_F+\delta k^\star -\abs{\mathbf{k}}} = \begin{cases}
		0 \quad\text{if}\quad &\abs{\mathbf{k}}<k_F \\
		1 \quad\text{if}\quad  k_F<\hspace{-0.6em}&\abs{\mathbf{k}}<k_F+\delta k^\star \\
		0 \quad\text{if}\quad  &\abs{\mathbf{k}}>k_F+\delta k^\star \\
	\end{cases}
\]
Then:
\[
	\lrR{\epsilon_\mathbf{k} + E^{(\mathrm{b})} - \epsilon_F} \alpha_\mathbf{k} = V_0 A\lrR{\mathbf{k}} \sum_{\mathbf{k}'} A\lrR{\mathbf{k}'} \alpha_{\mathbf{k}'}
\]
It follows:
\[
	\alpha_\mathbf{k} = \frac{V_0 A\lrR{\mathbf{k}}}{\epsilon_\mathbf{k} + E^{(\mathrm{b})} - \epsilon_F} \sum_{\mathbf{k}'} A\lrR{\mathbf{k}'} \alpha_{\mathbf{k}'}
\]
We multiply both sides by $A\lrR{\mathbf{k}}$ and sum over $\mathbf{k}$. The coefficient $\sum A \alpha$ can be eliminated by simplification, leaving the self-consistency equation
\[
	\sum_\mathbf{k} \frac{V_0 A\lrR{\mathbf{k}}}{\epsilon_\mathbf{k} + E^{(\mathrm{b})} - \epsilon_F} = 1
\]

We define $\kappa$ as in the previous section, $\kappa \equiv (2\pi)^D/ L^D V_0$, and convert the sum in an integral,
\[
	\kappa = \int_{\abs{\mathbf{k}}<\Lambda} d^D \mathbf{k} \, \frac{A\lrR{\mathbf{k}}}{\epsilon_\mathbf{k} + E^{(\mathrm{b})} - \epsilon_F}
\]
Since $\delta k^\star \ll k_F$ by the assumption of thinness of the shell, to this one the integration domain is limited by the function $A$:
\[
	\kappa = \int_{\abs{\mathbf{k}}\ge k_F}^{\abs{\mathbf{k}}\le k_F+\delta k^\star} \frac{d^D \mathbf{k}}{\epsilon_\mathbf{k} + E^{(\mathrm{b})} - \epsilon_F}
\]
We make use of the $D$-dimensional density of states $\rho_D(\epsilon)$ to convert this to an energy integral,
\[
	\kappa = \int_{\epsilon_F}^{\epsilon_F+\delta\epsilon^\star} \frac{d\epsilon \, \rho_D(\epsilon)}{\epsilon + E^{(\mathrm{b})} - \epsilon_F}
\]
For $D=3$, the density of states depends on energy as $\sqrt{\epsilon}$, approximately horizontal around the Fermi energy. Then for any energy in the range of interest we can approximate $\rho_3 (\epsilon) \simeq \rho_3 (\epsilon_F) \equiv \rho_0$. It follows
\[
	\kappa \simeq \rho_0 \int_{\epsilon_F}^{\epsilon_F+\delta\epsilon^\star} \frac{d\epsilon}{\lrR{\epsilon-\epsilon_F} + E^{(\mathrm{b})}}
\]
This integral is analogous to the $D=1$ integral of the precedent section. Then for any given $\kappa$ a binding energy $E^{(\mathrm{b})}$ exists such that the above equation is satisfied. The integral can be solved, giving
\[
	\kappa \simeq \rho_0 \log \lrR{1+ \frac{\delta\epsilon^\star}{E^{(\mathrm{b})}}}
\]
Now: $\delta\epsilon^\star$ represents the maximum energy the pairing can take up from the pair, so in general $E^{(\mathrm{b})} < \delta\epsilon^\star$. It is reasonable to assume that low-lying excited states for which all this description works are formed near the Fermi surface, such that $E^{(\mathrm{b})} \ll \delta\epsilon^\star$. Then
\[
	\frac{\kappa}{\rho_0} \simeq \log \lrR{\frac{\delta\epsilon^\star}{E^{(\mathrm{b})}}}
	\quad\implies\quad
	E^{(\mathrm{b})} \simeq \delta\epsilon^\star e^{-\eta}
\]
where $\eta \equiv \kappa/\rho_0 = (2\pi)^3 / L^3 V_0 \rho_0$. Then the strength of the binding is given by the energy extension of the interaction shell, suppressed exponentially by a factor $\eta \propto V_0^{-1}$. This makes sense: strong interactions produce negligible damping, and the strength of the binding is controlled by how much the shell is thick. On the contrary, weak interactions produce a strong damping, making it much difficult for the shell thickness to compensate. One can think about the shell width as a measure of \textit{how many} states can couple.

The whole argument holds for $D=2$, for which the density of states is a constant, and $D=1$, for which it goes like $\epsilon^{-1/2}$. This section lets us conclude that a ``shell interaction'' of strength $V_0$ creates electron pairs quite independently of $V_0$, as long as it is not too small. This is an astonishing result: not only it effectively corrects the incoherence of the above section, but it also demonstrates that the pairing of electrons observed in superconductors is a collective phenomenon arising from the presence of an entire electron liquid.

Now, the next step is to understand how this interaction comes to life at all. We know we are inside a material, a crystal of some kind: it is necessary to screen the Coulomb interaction. It is natural to look for any kind of \textit{attraction} inside the interactions of electrons with the crystal, instead of interactions of electrons with themselves. Moreover, we need some kind of quantum mechanism capable of storing the binding energy of the pair - which we now start calling a \textbf{Cooper pair}. As the electromagnetic field stores the binding energy of an atom with its electrons (pictorially we say they ``exchange a photon'', although this description is quite misleading), we expect some quantized field of the material to mediate the interaction and store the binding energy of the Cooper pair. In general we may look for any kind of collective excitation of materials -- quasiparticles of any kind -- but the most general, simple and obvious is the phonon, ``the quantum of lattice vibrations''.

\section{The role of phonons in superconductivity}

The phononic field is quantized in crystals: a good source about such quantization is \citetitle{grosso2000solid} \cite{grosso2000solid} by \citeauthor{grosso2000solid}; we will just rapidly sketch the essential concepts.
Consider a general crystal with $n$ atoms per unit cell. The index $\lambda = 1,\cdots,n$ varies over the cell atoms. The ion in position $\lambda$ has mass $M_\lambda$. The hamiltonian describing phonons is an harmonic one, with dispersion $\Omega_{\mathbf{k}\nu}$
\begin{equation}\label{eq:born-oppenheimer phonon hamiltonian}
	\hat H^{(\mathrm{p})} = \sum_\mathbf{k} \sum_\nu  \hbar \Omega_{\mathbf{k}\nu} \lrS{\hat a_{\mathbf{k}\nu}^\dagger \hat a_{\mathbf{k}\nu} + \frac{\mathbb{1}}{2}}
\end{equation}
with $\hat a_{\mathbf{k}\nu}^\dagger$ the creation operator for an oscillation of wavevector $\mathbf{k}$ in band $\nu$, and $\hat a_{\mathbf{k}\nu}$ the related destruction operator.  Such operators obey Bose commutation rules:
\[
	\comm{\hat a_{\mathbf{k}\nu}}{ \hat a_{\mathbf{h}\mu}^\dagger} = \delta_{\mathbf{k}\mathbf{h}}
	\delta_{\nu\mu}
\]
The next section is devoted to a very rapid description of how this hamiltonian is obtained.

\subsection{Born-Oppenheimer quantization}

In order to recover the hamiltonian in Eq.~\eqref{eq:born-oppenheimer phonon hamiltonian}, \textbf{Born-Oppenheimer approximation} is used: the motion of ions is assumed to be much slower than those of electrons, thus their positions $\mathbf{R}_{i\lambda}$ (with $i$ the index for the $i$-th cell and $\lambda$ the index for the atom inside the cell) are assumed as a parameter for the electronic problem and this last is solved, giving out a parametric solution for the energy $E\lrS{\mathbf{R}}$. Here $\mathbf{R}$ represents the vector collecting all vectors $\mathbf{R}_{i\lambda}$.

Throughout a variational approach over the parametric solution $E\lrS{\mathbf{R}}$ around a energy minimum (a certain ions configuration $\mathbf{
R}^{(0)}$) the energy is reduced to a quadratic expression in terms of the ions displacement from rest positions, $\mathbf{u}_{i\lambda} \equiv \mathbf{R}_{i\lambda} - \mathbf{R}_{i\lambda}^{(0)}$. Including also the kinetic contribution, the overall result is
\[
	\hat H^{(\mathrm{p})} = \sum_i \sum_\lambda
	\lrB{
		\frac{\abs{\hat{\mathbf{p}}_{i\lambda}}^2}{2M_\lambda}
		+ \half \sqrt{M_\lambda} \hat{\mathbf{u}}_{i\lambda} \cdot \sum_j \sum_\eta 
		\lrS{
			\frac{1}{\sqrt{M_\lambda M_\eta}}
			\pdv[2]{E\lrS{\mathbf{R}}}{\mathbf{R}_{i\lambda}}{\mathbf{R}_{j\eta}}
		}
		\cdot
		\sqrt{M_\eta} \hat{\mathbf{u}}_{j\eta}
	}
\]
The mass factors are included to simplify the following steps.
Now: performing a Discrete Fourier Transform over the above equation and recognizing the dynamical matrix $\mathcal{D}$ in the (transformed) matrix in the quadratic term, one gets
\[
	\hat H^{(\mathrm{p})} = \sum_\mathbf{k} \sum_\lambda \lrB{
		\frac{\hat{\mathbf{p}}_{\mathbf{k}\lambda} \cdot \hat{\mathbf{p}}_{-\mathbf{k}\lambda}}{2M_\lambda}
		+ \half \sqrt{M_\lambda} \hat{\mathbf{u}}_{\mathbf{k}\lambda}\cdot \sum_\eta 
		\mathcal{D}_{\lambda\eta}\lrR{\mathbf{k}}
		\cdot
		\sqrt{M_\eta} \hat{\mathbf{u}}_{-\mathbf{k}\eta}
	}
\]
Diagonalization of $\mathcal{D}$ is possible. Such diagonalization is intended over the sublattice index $\lambda$, and provides a set of bands parameterized by the index $\nu=1,\cdots,n$, 
\[
	\mathcal{D}\lrR{\mathbf{k}} \mathbf{w}_{\mathbf{k}}^{(\nu)} = \Omega_{\mathbf{k}\nu}^2 \mathbf{w}_{\mathbf{k}}^{(\nu)}
\]
The eigenvectors
\[
	\lrS{\mathbf{w}_{\mathbf{k}}^{(\nu)}}_\lambda
\]
are $n$ in number, and each one has $n$ components and is a specific mixed oscillations of the sublattices. The eigenvalues at fixed wavevector $\Omega_{\mathbf{k}\nu}^2$ are also $n$ in general. 
Now: define $\mathrm{U}_{\mathbf{k}\nu}$ such that
\[
	\mathrm{U}_{\mathbf{k}\nu} \equiv
	\sum_\lambda
	\mathbf{u}_{\mathbf{k}\lambda} \cdot
	\mathbf{w}_{\mathbf{k}\lambda}^{(\nu)}
\]
Remember: two scalar products are involved here. One is the spatial one, indicated by ``$\cdot$``. The other is the $\lambda$ scalar product, indicated by the sum and the index contraction. Then $\mathrm{U}_{\mathbf{k}\nu}$ is defined as the projection of the vector $\mathbf{u}_\mathbf{k}$ (the vector which has as components the displacement of the $\lambda$-th sublattice) onto the eigenvector $\mathbf{w}_{\mathbf{k}}^{(\nu)}$. It follows obviously
\begin{equation}\label{eq:bcs-definition of U}
	\mathbf{u}_\mathbf{k} = \sum_\nu \mathrm{U}_{\mathbf{k}\nu} \mathbf{w}_{\mathbf{k}}^{(\nu)}
\end{equation}
Analogously
\[
	\mathrm{P}_{\mathbf{k}\nu} \equiv
	\sum_\lambda
	\mathbf{p}_{\mathbf{k}\lambda} \cdot
	\mathbf{w}_{\mathbf{k}\lambda}^{(\nu)}
	\quad\implies\quad
	\mathbf{p}_\mathbf{k} = \sum_\nu \mathrm{P}_{\mathbf{k}\nu} \mathbf{w}_{\mathbf{k}}^{(\nu)}
\]
Using orthonormality of the eigenvectors one gets by substitution
\[
	\hat H^{(\mathrm{p})} = \sum_\mathbf{k} \sum_\lambda \lrB{
		\frac{\hat{\mathrm{P}}_{\mathbf{k}\lambda} \hat{\mathrm{P}}_{-\mathbf{k}\lambda}}{2M_\lambda}
		+ \half M_\lambda \Omega_{\mathbf{k}\lambda}^2 \hat{\mathrm{U}}_{\mathbf{k}\lambda} \hat{\mathrm{U}}_{-\mathbf{k}\lambda}
	}
\]

This one is the hamiltonian of $n$ independent harmonic oscillators. Each oscillator has a specific frequency $\Omega_{\mathbf{k}\lambda}$ at fixed wavevector. Then an expression for the displacements and momenta in terms of the Bose operators is possible,
\[
	\hat{\mathrm{U}}_{\mathbf{k}\lambda} = \sqrt{\frac{\hbar}{2M_\lambda \Omega_{\mathbf{k}\lambda}}} \lrR{\hat{a}_{\mathbf{k}\lambda}^\dagger + \hat{a}_{\mathbf{k}\lambda}}
	\qquad
	\hat{\mathrm{P}}_{\mathbf{k}\lambda} = \sqrt{\frac{\hbar}{2M_\lambda \Omega_{\mathbf{k}\lambda}}} \lrR{\hat{a}_{\mathbf{k}\lambda}^\dagger - \hat{a}_{\mathbf{k}\lambda}}
\]
It is of our particular interest the expression for the three-dimensional displacement, thus we reconstruct $\mathbf{u}_\mathbf{k}$ through Eq.~\eqref{eq:bcs-definition of U},
\[
	\hat{\mathbf{u}}_\mathbf{k} = \sum_\lambda \hat{\mathrm{U}}_{\mathbf{k}\lambda} \mathbf{w}_{\mathbf{k}}^{(\lambda)} = \sum_\lambda \sqrt{\frac{\hbar}{2M_\lambda \Omega_{\mathbf{k}\lambda}}} \lrR{\hat{a}_{\mathbf{k}\lambda}^\dagger + \hat{a}_{\mathbf{k}\lambda}} \mathbf{w}_{\mathbf{k}}^{(\lambda)}
\]
Notice that for simple crystals ($n=1$) the $\lambda$ sum vanishes. This will be applied later on.

\subsection{Electron-phonon interactions}

The key idea to model electron-phonon interactions is that ions do not get ``too far'' from the equilibrium positions $\mathbf{R}_{i\lambda}^{(0)}$. Then, coherently with Born-Oppenheimer approximation, the total hamiltonian
\[
	\hat H = \hat H^{(\mathrm{e})} + \hat H^{(\mathrm{ep})} + \hat H^{(\mathrm{p})}
\]
contains an interaction term $H^{(\mathrm{ep})}$ which depends on the electron positions $\mathbf{r}_\alpha$ and the ions positions $\mathbf{R}_{i\lambda}$ and can be expanded as
\[
	\hat H^{(\mathrm{ep})} \lrR{\mathbf{r},\mathbf{R}} \simeq \hat H^{(\mathrm{ep})} \lrR{\mathbf{r},\mathbf{R}^{(0)}} + \sum_i \grad_{\mathbf{R}_{i\lambda}} \hat{H}^{(\mathrm{ep})} \big|_{\mathbf{R}_{i\lambda} = \mathbf{R}_{i\lambda}^{(0)}} \cdot \hat{\mathbf{u}}_{i\lambda} + \cdots
\]
The above expression includes phonon operators and electron operators, both indicated by the ``hat''. We already have $\hat{\mathbf{u}}_{i\lambda}$ from last section. The first term is identically null, because the contribution arising from rest positions of the ions is already included in the bare phonon hamiltonian. The interaction is Coulomb-like, thus
\[
	\grad_{\mathbf{R}_{i\lambda}} \hat{H}^{(\mathrm{ep})}\big|_{\mathbf{R}_{i\lambda} = \mathbf{R}_{i\lambda}^{(0)}} = \grad_{\mathbf{R}_{i\lambda}} \sum_\alpha v \lrR{\abs{\hat{\mathbf{r}}_\alpha - \hat{\mathbf{R}}_{i\lambda}^{(0)}}}
	\quad\qq{with}\quad
	v(r) = -\frac{Ze^2}{r}	
\]
Moving to a field formalism, this single-particle operator becomes
\[
	\sum_\alpha v \lrR{\abs{\hat{\mathbf{r}}_\alpha - \mathbf{R}_{i\lambda}^{(0)}}} \to \int_{\R^D} d^D \mathbf{r} \, \hat{\psi}^\dagger (\mathbf{r}) v \lrR{\abs{\mathbf{r} - \mathbf{R}_{i\lambda}^{(0)}}} \hat{\psi} (\mathbf{r})
\]
where $\hat{\psi}$ is the electronic field operator, acting on the electrons Hilbert space. The algebraic notation ``$\dagger$'' now is intended upon such space. Such field can be expanded through a complete set of single particle Bloch spin-wavefunctions,
\[
	\hat\psi(\mathbf{r}) = \sum_{\mathbf{k}\sigma} \sum_\lambda \varphi_{\mathbf{k}\lambda}(\mathbf{r}) \hat c_{\mathbf{k}\sigma\lambda}
	\quad\qq{with}\quad
	\varphi_{\mathbf{k}\lambda}(\mathbf{r}) = \frac{1}{L^{D/2}} u_{\mathbf{k}\lambda} (\mathbf{r}) e^{i \mathbf{k} \cdot \mathbf{r}}
\]
and $u$ a lattice-periodic function.
For simplicity we consider simple crystals, for which $n=1$. For such crystals the dispersion of phonons has an energy extension of approximately $\hbar\omega_D$, with $\omega_D$ the Debye frequency; in the language of the Sec.~\ref{subsec:bound states, considering statistics}, $\delta\epsilon^\star = \hbar\omega_D$. For composite crystals, due to the presence of optical bands and polarization effects, the argument must be corrected (sometimes, fatally).

A long and tedious calculation no one can convince the author to perform leads the rather simple result
\begin{equation}\label{eq:H-ep final form}
	\hat H^{(\mathrm{ep})} = \sum_{\mathbf{k}\sigma} \sum_\mathbf{q} g_\mathbf{k}\lrR{\mathbf{q}} \hat{c}_{\mathbf{k}+\mathbf{q}\sigma}^\dagger \hat{c}_{\mathbf{k}\sigma} \lrR{\hat{a}_\mathbf{q}^\dagger + \hat{a}_\mathbf{q}}
\end{equation}
with
\[
	g_\mathbf{k}\lrR{\mathbf{q}} \equiv \lrR{-i\mathbf{q} \cdot \mathbf{w}_\mathbf{q} v_\mathbf{q}} \sqrt{\frac{\hbar}{2 N M \Omega_\mathbf{q}}} \ev{u_{\mathbf{k}+\mathbf{q}}^*(\mathbf{r}) u_\mathbf{k}(\mathbf{r})}_c
\]
where $v_\mathbf{q}$ is the $\mathbf{q}$ component of the Coulomb potential, $N$ is the number of cells and $\ev{\cdots}_c$ is the spatial mean over a single cell. Aside from how $g$ turns out to be, we are interested in the general form of Eq.~\eqref{eq:H-ep final form}. It is made of two contributions: the first contribution describes the scattering of one electron with creation of a phonon with wavevector $\mathbf{q}$; the second is the same process, but with absorption of one phonon of same wavevector. Both have amplitude $g_\mathbf{k}\lrR{\mathbf{q}}$.