\section{The Abrikosov lattice}

So, in certain conditions magnetic flux can penetrate the sample in a quantized manner, piercing holes with a specific field-current structure -- the Abrikosov vortices. Each of this vortices carries exactly one quantum of flux $\Phi_0$, its ``charge''. Clearly, inverting the direction of the penetrating field, the sign of the charge inverts. Then two classes of vortices exist -- those with charge $+\Phi_0$ and those with charge $-\Phi_0$. The noun ``charge'' is all but casual: as it turns out, \textbf{vortices with same charge repel each other, while vortices with same charge attract each other}. Impressive, isn't it?

This section is devoted to the study of the so-called Abrikosov lattice. The phenomenology is the following: when many vortices with the same charge nucleate in the superconducting sample (maintaining low the vortices density) they create a regular configuration -- a triangular lattice. These vortices, also called \textbf{fluxons}, behave just like particles with a certain interaction law. As low-energy excited states, they are quasiparticles. First, we take off by studying how two fluxons interact.

\subsection{Two static fluxons}

\begin{figure}
    \centering
    \begin{tikzpicture}[
    cube/.style={
        color=lev!30
    }]
    \def\xlength{2}
    \def\ylength{1.7}
    \def\zlength{1}
    \begin{axis}[
        axis x line=center,
        axis y line=center,
        axis z line=center,
        axis on top,
        axis equal image,
        xlabel={$x$},
        ylabel={$y$},
        zlabel=\empty,
        xlabel style={right},
        ylabel style={above},
        zlabel style=\empty,
        xtick=\empty,
        ytick=\empty,
        ytick=\empty,
        xticklabels=\empty,
        yticklabel=\empty,
        zticklabel=\empty,
        xmin=-\xlength*1.1, xmax=\xlength*1.1,
        ymin=-\ylength*1.1, ymax=\ylength*1.1,
        zmin=-\zlength*1.1, zmax=\zlength*1.1,
        view={0}{90}]

        \newcommand{\modulation}[3]{
            e^( 
                -( (x-#1)^2 + (y-#2)^2 ) / #3 % Exponential damping
            ) / sqrt(
                (x-#1)^2 + (y-#2)^2 % Normalization
            )
        }

        \def\AfluxonX{-1.22}
        \def\AfluxonY{-0.64}
        \def\BfluxonX{1.3}
        \def\BfluxonY{1.1}
        \def\damping{0.7}

        \shade[outer color=white,inner color=lev!30]
            (axis cs: \AfluxonX,\AfluxonY) circle (\damping*1.5);
        \shade[outer color=white,inner color=lev!30]
            (axis cs: \BfluxonX,\BfluxonY) circle (\damping*1.5);

        \addplot3[
            color=lev!60,
            domain=-\xlength:\xlength,
            domain y=-\ylength:\ylength,
            quiver={
                u={
                    -(y-\AfluxonY)*\modulation{\AfluxonX}{\AfluxonY}{\damping} +
                    -(y-\BfluxonY)*\modulation{\BfluxonX}{\BfluxonY}{\damping}
                },
                v={
                    (x-\AfluxonX)*\modulation{\AfluxonX}{\AfluxonY}{\damping} +
                    (x-\BfluxonX)*\modulation{\BfluxonX}{\BfluxonY}{\damping}
                },
                scale arrows=0.17
            },
            -stealth,
            samples=16
        ] {0};
    \end{axis}
\end{tikzpicture}
    \caption{Pictorial representation of two ``distant'' fluxons. The arrows on the $xy$ represent current flows, while the shading (of radius $\sim\lambda$) represents field intensity.}
    \label{fig:two fluxons interacting}
\end{figure}

