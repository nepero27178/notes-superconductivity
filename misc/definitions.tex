% --- Ambiente citazione ---

    \newenvironment{cit}[2]
        {
            \begin{mdframed}[
                linewidth=1pt,
                linecolor=lev!50,
                leftmargin=1em,
                bottomline=false,
                topline=false,
                rightline=false,
                innerrightmargin=0pt,
                innertopmargin=0pt,
                innerbottommargin=0pt,
                innerleftmargin=1em,
                skipabove=1em,
                skipbelow=1em
            ]
            \noindent\normalfont   
            \textbf{\citename{#1}{author}}: \textbf{\citetitle{#1}}  (\cite{#1} @ #2)
            \vspace{0.5em}\newline
        }
        {\end{mdframed}}

    
% --- Comandi matematici utili ---
    \def\beq{\begin{equation}}
    \def\eeq{\end{equation}}
    % Numerelli
        \newcommand{\C}{\mathbb{C}}
        \newcommand{\R}{\mathbb{R}}
        \newcommand{\N}{\mathbb{N}}
        \newcommand{\Z}{\mathbb{Z}}
    % Fisica
        \newcommand{\ham}{\hat{H}}
        \newcommand{\hilb}{\mathcal{H}}
        \newcommand{\vphi}{\varphi}
        \newcommand{\normalorder}[1]{\colon #1 \, \colon}
    % Algebra
        % \newcommand{\Id}{\mathbb{1}}[1]{\ensuremath{\mathbf{\bar{#1}}}}
    % Altro
        \newcommand{\nquad}{\mkern-18mu}              % deindentazione
        \newcommand{\sgn}{\mathrm{sgn}}               % funzione segno
        \newcommand{\imp}{\stackrel{!}{=}}            % imposizione
        \newcommand{\questionmarkeq}{\stackrel{?}{=}} % uguale con ?
        \newcommand{\half}{%
          \ifmmode
            \frac{1}{2} % \half in mathmode
          \else
            $1/2$ % \half in text
          \fi
        }
        \newcommand{\ol}[1]{\overline{{#1}}}
        \newcommand{\lrR}[1]{\left({#1}\right)} % left and right round parentheses
        \newcommand{\lrS}[1]{\left[{#1}\right]} % left and right square parentheses
        \newcommand{\lrB}[1]{\left\{{#1}\right\}} % left and right braces
        \newcommand{\dV}[1] {d^3 \mathbf{#1}\,} % elemento di volume        
    % Versors
        \newcommand{\versor}[1]{\ensuremath{\mathbf{\bar{#1}}}}
    
% --- Simbolo per sezioni/sottosezioni Opzionali ---
    \newcommand{\optional}{\texorpdfstring{\textreferencemark{} }{※ }} 

% --- EqBox ---
    % Pacchetto di riferimento: tcolorbox
    \newtcolorbox{eqbox}[1][]{
      colback=lev!15,
      top=0.4em,     
      bottom=0.8em,
      colframe=white,
      width=\textwidth,
      after={\par\smallskip\noindent\ignorespacesafterend}, %
      #1 % parametro (placeholder) per eventuali
         % argomenti facoltativi
    }

% --- Midline arrow
\newcommand{\arrowIn}{
    \tikz \draw[-stealth] (-1pt,0) -- (1pt,0);
}

% --- Indice locale ---
    % Pacchetto di riferimento: etoc
    \newcommand{\chaptertoc}[1][]{%
      \etocsettocstyle{\addsec*{#1}}{}%
      \localtableofcontents%
      \vspace{1cm}
    }